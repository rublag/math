\RequirePackage[l2tabu,orthodox]{nag}
\documentclass[a4paper, fleqn]{report}
%\documentclass[a4paper]{report}
\usepackage[margin=0.5cm,includefoot,heightrounded]{geometry}
\usepackage[intlimits]{amsmath}
\usepackage{amssymb}
\usepackage{bm}
\usepackage{amsthm}
\usepackage{mathtools}
\usepackage[colorlinks,unicode]{hyperref}
\usepackage[T1,T2A]{fontenc}
\usepackage[utf8]{inputenc}
\usepackage[russian]{babel}
\usepackage{titlesec}
\usepackage{layout}
\usepackage{dsfont}
\usepackage{tikz}
\usepackage{multirow}
\setlength{\mathindent}{0pt}

\pagestyle{plain}

\titleformat{\chapter}[block]{\normalfont\huge\bfseries\centering}{\S \thechapter.}{20pt}{\Huge}
\newcommand*{\CC}{\mathbb{C}}
\newcommand*{\RR}{\mathds{R}}
\newcommand*{\ZZ}{\mathds{Z}}
\newcommand*{\NN}{\mathds{N}}
\newcommand*{\QQ}{\mathds{Q}}

\newcommand*{\EV}{\mathbf{E}}
\newcommand*{\Var}{\mathbf{D}}
\newcommand*{\PR}{\mathbf{P}}

\newcommand*{\D}{\bm{D}}
\newcommand*{\R}{\bm{R}}
\newcommand{\BraceThree}{%
	\multirow{3}{*}{$\left\{\begin{array}{@{}l@{}} \null \\ \null \\ \null\end{array}\right.$}%
}

\newcommand{\BraceTwo}{%
	\multirow{2}{*}{$\left\{\begin{array}{@{}l@{}} \null \\ \null\end{array}\right.$}%
}

\renewcommand\qedsymbol{$\blacksquare$}

\theoremstyle{plain}
\newtheorem{Th}{Теорема}
\newtheorem*{Th*}{Теорема}
\newtheorem{Lem}{Лемма}
\newtheorem*{Lem*}{Лемма}
\newtheorem*{St}{Утверждение}
\newtheorem*{Prop}{Свойства}

\theoremstyle{definition}
\newtheorem*{Def}{Определение}
\newtheorem*{Ex}{Пример}
\newtheorem*{Nam}{Обозначение}
\newtheorem*{Agr}{Договоренность}

\theoremstyle{remark}
\newtheorem*{Rem}{Замечание}
\newtheorem*{Probl}{Упражнение}

\renewcommand{\proofname}{Доказательство}
\addto\captionsrussian{
	\renewcommand{\proofname}{Доказательство}
}

\newenvironment{amatrix}[1]{%
	\left(\begin{array}{@{}*{#1}{c}|c@{}}
	}{%
	\end{array}\right)
}

\DeclareMathOperator*{\Res}{Res}
\DeclareMathOperator{\PE}{E}
\DeclareMathOperator{\PP}{P}

\renewcommand{\Re}{\operatorname{Re}}
\renewcommand{\Im}{\operatorname{Im}}



\begin{document}
\section {Лемма Жордана (с другим контуром)}
В обычной лемме Жордана используется контур $Re^{i\pi t}$ с $t \in [0, 1]$. Нам нужен будет контур $\alpha + Re^{i t \pi/2}$ c $t \in [-1, 1]$
\begin{Th*}[Лемма Жордана]
	$F(s)$ непрерывна в области $G = \{s | Res \le  \alpha, |z-\alpha| \ge  R_0 > 0\}$ \\
	$C_R$ --- полуокружность $|z-\alpha| = R$ в $G$ \\
	$\lim_{R \to \infty} \sup_{s \in C_R} |F(s)| = 0$ \\
	$t > 0$
	Тогда $\lim_{R \to \infty} \int_{C_R} F(s)e^{st}ds = 0$
\end{Th*}
\begin{proof} $ $\\
По условию, $\forall \varepsilon > 0 \ \  \exists R \ \ \forall s \in C_R \quad \left| F(s) \right| = \left| F\left(\alpha + Re^{i\varphi}\right) \right|  < \varepsilon$, тогда

\begin{align*}
    & \left|\int_{C_R} e^{st} F(s) ds\right| \le
      \int_{C_R} \left|e^{st} F(s)\right| ds \le
      \varepsilon \int_{C_R} \left|e^{st}\right| ds  \\
    & \text{На $C_R$: } \left|e^{st}\right| = \left|e^{\left(\alpha+R\cos\varphi+Ri\sin\varphi\right)t}\right| = e^{\left(\alpha + R\cos\varphi\right) t} \\
    & \text{Отсюда } \varepsilon \int_{C_R} \left|e^{st}\right| ds = 
      \varepsilon \int_{\frac{\pi}{2}}^{\frac{3\pi}{2}} e^{\alpha t + R t \cos \varphi} dRe^{i\varphi} = 
      \varepsilon \int_{\frac{\pi}{2}}^{\frac{3\pi}{2}} e^{\alpha t + R t \cos \varphi} R i e^{i\varphi} d\varphi \le  \\
    & \varepsilon \int_{\frac{\pi}{2}}^{\frac{3\pi}{2}} \left|e^{\alpha t + R t \cos \varphi} R i e^{i\varphi} \right| d\varphi = 
      R\varepsilon e^{\alpha t} \int_{\frac{\pi}{2}}^{\frac{3\pi}{2}} e^{R t \cos\varphi} d\varphi = 
      R\varepsilon e^{\alpha t} \int_{0}^{\frac{\pi}{2}} e^{R t \cos \left(\varphi + \frac{pi}{2}\right)} d\varphi = \\
    & R\varepsilon e^{\alpha t} \int_{0}^{\pi} e^{-R t \sin\varphi} d\varphi = 
      2R\varepsilon e^{\alpha t} \int_{0}^{\frac{\pi}{2}} e^{-R t \sin\varphi} d\varphi \\
    & \text{На $[0, \frac{\pi}{2}]$ } \sin\varphi \ge \frac{2}{\pi} \varphi \\
    & \implies 2R\varepsilon e^{\alpha t} \int_{0}^{\frac{\pi}{2}} e^{-R t \sin\varphi} d\varphi \le
      2R\varepsilon e^{\alpha t} \int_{0}^{\frac{\pi}{2}} e^{-R t \frac{2}{\pi}\varphi} d\varphi = \\
    & 2R\varepsilon e^{\alpha t} \left.\left( \frac{1}{-\frac{2Rt}{\pi}}e^{- \frac{2 R t \varphi}{\pi}} \right) \right|_0^{\frac{pi}{2}} = 
         \frac{\pi\varepsilon}{t} e^{\alpha t} \left(1 - e^{- R t} \right) \xrightarrow[R \to \infty]{} 0
\end{align*}

Отсюда интеграл по дуге стремится к $0$.
\end{proof}

\section{Замена переменной в обратном преобразовании Лапласа}
\begin{Th*}
\[
    L^{-1}_s \left[ f(cs) \right](z) = L^{-1}_s \left[ \frac{1}{c}f(s) \right]\left(\frac{z}{c}\right)
\]
\end{Th*}
\begin{proof}
\begin{align*}
& L^{-1}_s\left[f(cs)\right](z) 
= \int_{\alpha-i\infty}^{\alpha+i\infty} e^{sz}f(cs)ds 
= \frac{1}{c} \int_{\alpha-i\infty}^{\alpha+i\infty} e^{csz/c} f(cs) dcs
= \frac{1}{c} \int_{\alpha-i\infty}^{\alpha+i\infty} e^{sz/c} f(s) ds
= L^{-1}_s \left[ \frac{1}{c} f(s) \right]\left( \frac{z}{c} \right)  
\end{align*}
\end{proof}

\section {Сводим задачу к преобразованию Лапласа}
Есть случайные величины $X, Y, Z$. Мы не знаем распределение $X$, знаем распределение $Y$ и наблюдаем $Z$. Кроме того, известно, что $Z = XY$, и что все величины непрерывны. Нужно оценить распределение $X$.

Мы будем использовать вейвлет «Mexican hat», потому что он прост и непрерывен. \\
$\psi(t) = \frac{2}{\sqrt{3} pi^{1 / 4}} (1-t^2)e^{-t^2 / 2}$ --- его формула. \\

Кроме того, определим элементы фрейма: \\
$\psi_{m,n}(t) = \frac{1}{2^m} \psi\left( \frac{t}{2^m}-n \right)  \frac{1}{\sqrt{2^m} }\frac{2}{\sqrt{3} pi^{1 / 4}} (1-\left( \frac{t}{2^m} - n \right)^2)e^{-\left( \frac{t}{2^m} - n \right)^2 / 2}$ --- его формула. \\

Рассмотрим случай, когда $Y \sim \chi^2_{2k}$. \\
$\chi^2_{2k} \sim \frac{1}{2^k} \frac{1}{\Gamma(k)} x^{k-1} e^{-x / 2}$ -- плотность $\chi^2_{2k}$ \\


Рассмотрим $\PE g(Z)$ для некоторой функции $g$.
\begin{align*}
    & \PE g(Z) = \PE g(XY) = \int_{-\infty}^{\infty} \int_{-\infty}^{\infty} g(xy) f_X(x) f_Y(y) dx dy \\
    & \PE \psi_{m,n}(X) = \int_{-\infty}^{\infty} \psi_{m,n}(x) f_X(x) dx 
\end{align*}

Будем строить функции $g_{m,n}$ такие, что $\PE g_{m,n}(Z) = \PE \psi_{m,n}(X)$. Заметим, что достаточно выполнения:
\[
    \forall x \in \Im X \quad \int_{-\infty}^{\infty} g_{m,n}(xy)f_Y(y)dy = \psi_{m,n}(x) 
\]

Перепишем левую часть (и будем писать $g$ вместо $g_{m,n}$):
\begin{align*}
    & \int_{-\infty}^{\infty} g(xy) f_Y(y) dy = 
      \int_{0}^{\infty} g(xy) \frac{1}{\Gamma(k)} \frac{1}{2^k} y^{k-1} e^{-y/2} dy = \\
    & \int_{0}^{\infty} g(z) \frac{1}{2^k} \frac{1}{\Gamma(k)} \left( \frac{z}{x} \right)^{k-1} e^{-z / 2x} \frac{dz}{x} =
      \frac{1}{x^k} \frac{1}{2^k} \frac{1}{\Gamma(k)} \int_{0}^{\infty} g(z) z^{k-1} e^{-zu} dz \\
    & \text{Введем замену $u = \frac{1}{2x}$: }\\
    & \frac{1}{x^k} \frac{1}{2^k} \frac{1}{\Gamma(k)} \int_{0}^{\infty} g(z) z^{k-1} e^{-z / 2x} dz =
      u^k \frac{1}{\Gamma(k)} \int_{0}^{\infty} g(z) z^{k-1} e^{-zu} dz = \\
    & u^k \frac{1}{\Gamma(k)} L\left[ g(z)z^{k-1} \right](u) 
\end{align*}
В правой части:
\[
    \psi_{m,n}(x) =
    \psi_{m,n}\left( \frac{1}{2u} \right) =
    \left( \frac{1}{\sqrt{2} } \right)^m \psi\left(\frac{1}{u} \frac{1}{2^{m+1}} - n\right)
\] 
Отсюда наше уравнение:
\[
    u^k \frac{1}{\Gamma(k)} L\left[ g(z)z^{k-1} \right](u) = 
    \left(\frac{1}{\sqrt{2}}\right)^m \psi\left(\frac{1}{u} \frac{1}{2^{m+1}}-n\right)
\] 
Начнем решать.
\begin{align*}
    & \implies L\left[ g(z) z^{k-1} \right](u) = \left( \frac{1}{\sqrt{2} } \right)^m \Gamma(k) \frac{1}{u^k} \psi \left( \frac{1}{u} \frac{1}{2^{m+1}} - n \right) \\
    & \implies g(z) z^{k-1} = L^{-1}_u \left[ \frac{1}{\sqrt{2^m} } \Gamma(k) \frac{1}{u^k} \psi \left( \frac{1}{u} \frac{1}{2^{m+1}} - n \right)  \right] (z) \\
    & \implies g(z) z^{k-1} \frac{1}{\Gamma(k)} \sqrt{2^m} = L^{-1}_u \left[ \frac{1}{u^k} \psi \left( \frac{1}{u} \frac{1}{2^{m+1}} - n \right) \right] (z) \\
    & \implies g(z) z^{k-1} \frac{1}{\Gamma(k)} \sqrt{2^m} = L^{-1}_u \left[ \frac{1}{u^k} \frac{2}{\sqrt{3} \pi^{1 /4}} \left( 1 - \left( \frac{1}{u} \frac{1}{2^{m+1}} - n \right)^2  \right) \exp \left[ -\frac{1}{2} \left( \frac{1}{2^{m+1}} \frac{1}{u} - n \right)^2  \right]   \right](z) \\
    & \implies g(z) z^{k-1} \frac{1}{\Gamma(k)} \sqrt{2^m} \frac{\sqrt{3} \pi^{1 /4}}{2} = L^{-1}_u \left[ \frac{1}{u^k} \left( \left( 1 - n^2 \right)  + \left( \frac{n}{u} \frac{1}{2^{m}}\right) - \left( \frac{1}{u^2} \frac{1}{2^{2m+2}} \right) \right) \exp \left[ -\frac{1}{2} \left( \frac{1}{u^2} \frac{1}{4^{m+1}} -  \frac{1}{u} \frac{n}{2^m} + n^2 \right)  \right]   \right](z) \\
\end{align*}

\begin{align*}
\begin{split}
    \implies g(z) z^{k-1} \frac{1}{\Gamma(k)} \sqrt{2^m} \frac{\sqrt{3} \pi^{1 /4}}{2} e^{n^2 /2} ={}& (1-n^2) L^{-1}_u \left[ \frac{1}{u^k} \exp \left( -\frac{1}{2} \frac{1}{u^2} \frac{1}{4^{m+1}} \right) \exp \left( \frac{1}{u} \frac{n}{2^{m+1}} \right)  \right](z) \\
    +{}& \frac{n}{2^m} L^{-1}_u \left[ \frac{1}{u^{k+1}} \exp \left( -\frac{1}{2} \frac{1}{u^2} \frac{1}{4^{m+1}} \right) \exp \left( \frac{1}{u} \frac{n}{2^{m+1}} \right)   \right](z) \\ 
    -{}& \frac{1}{4^{m+1}} L^{-1}_u \left[ \frac{1}{u^{k+2}} \exp \left( -\frac{1}{2} \frac{1}{u^2} \frac{1}{4^{m+1}} \right) \exp \left( \frac{1}{u} \frac{n}{2^{m+1}} \right)   \right](z) 
\end{split}
\end{align*}

Отсюда видно, что $n = 0$ --- особый случай. Отложим его.

\section{Находим обратное преобразование Лапласа}
Найдем $L^{-1}_s \left[ \frac{1}{s^p} \exp \left[ -\frac{1}{2 s^2} \right] \exp \left( \frac{n}{s} \right)  \right](z)$. Из этого простыми преобразованиями можно будет получить все три необходимых нам обратных преобразования.

\begin{align*}
    & L^{-1}_s \left[ \frac{1}{s^p} \exp \left[ -\frac{1}{2 s^2} \right] \exp \left( \frac{n}{s} \right)  \right](z) = \\
    & \frac{1}{2\pi i} \int_{\alpha - i \infty}^{\alpha+i\infty} e^{sz} \frac{1}{s^p} e^{-1/(2s^2)} e^{n/s} ds = \\
\end{align*}

Берем контур $C$ = $C_1 + C_2$, где $C_1$ --- искомый, а $C_2$ --- дуга окружности (слева от $C_1$ с центром в $(\alpha, 0)$).
\begin{align*}
    & \left|F(s)\right| := \left|\frac{1}{s^p} \exp \left[ -\frac{1}{2 s^2} \right] \exp \left( \frac{n}{s} \right)\right| = \\
    & \left|\frac{1}{r^p e^{i\phi}} \exp\left[ -\frac{1}{2 r^2 e^{2i\phi}} \right] \exp \left( \frac{n}{r e^{i\phi}} \right)\right| = \\
    & \frac{1}{r^p} \exp\left[ -\frac{\cos 2\phi}{2 r^2} \right] \exp \left( \frac{n \cos\phi}{r} \right) \xrightarrow[s \to \infty]{} 0 
\end{align*}

Поэтому, из леммы Жордана, интеграл по $C_2$ стремится к нулю

Так как у $F(s)$ единственная особая точка --- $0$, можно взять любое $\alpha > 0$

Применим основную теорему о вычетах:

\begin{align*}
    & \frac{1}{2\pi i} \int_{\alpha - i \infty}^{\alpha+i\infty} e^{sz} \frac{1}{s^p} e^{-1/(2s^2)} e^{n/s} ds = \\
    & \frac{1}{2\pi i} (2\pi i) \underset{0}{\Res} \left(e^{sz} \frac{1}{s^p} e^{-1/(2s^2)} e^{n/s} \right) = \\
    & c_{-1} \text{, где $c_{-1}$ --- $-1$ член ряда Лорана для } e^{sz} \frac{1}{s^p} e^{-1/(2s^2)} e^{n/s} = \\
    & d_{(-1+p)} \text{, где $d_{(-1+p)}$ --- $(p-1)$-й член ряда Лорана для } e^{sz} e^{-1/(2s^2)} e^{n/s} = \\
\end{align*}

То есть, нам нужно найти $p-1$-й член ряда Лорана для $e^{sz} e^{-1/(2s^2)} e^{n/s}$.

Найдем сначала необходимые члены ряда для $e^{sz} e^{n/s}$.
\begin{align*}
    & e^{sz} e^{n / s} = \left(\sum_{q=0}^{\infty} \frac{\left(sz\right)^q}{q!} \right) \left( \sum_{q=0}^{\infty} \frac{\left( -\frac{n}{s} \right)^q}{q!}  \right) = \\
    & \sum_{q=0}^{\infty} s^q \sum_{r=q}^{\infty} \frac{z^r}{r!} \frac{n^{r-q}}{(r-q)!} + \sum_{q=-\infty}^{-1} s^q \sum_{r=-\infty}^{q} \frac{z^{r-q}}{(r-q)!} \frac{n^{r}}{r!} \\
\end{align*}

Так как $p-1 \ge 0$, а $e^{-1/2s^2}$ раскрывается в ряд только с отрицательными степенями, то сумма с отрицательными степенями нам не пригодится. Упростим получившийся ряд:

\begin{align*}
    & \sum_{q=0}^{\infty} s^q \sum_{r=q}^{\infty} \frac{z^r}{r!} \frac{n^{r-q}}{(r-q)!} = \\
    & \sum_{q=0}^{\infty} s^q \sum_{r=q}^{\infty} \frac{\left( \frac{zn}{n} \right)^r}{r!} \frac{n^{r-q}}{(r-q)!} = \\
    & \sum_{q=0}^{\infty} s^q \sum_{r=q}^{\infty} \frac{\left(zn\right)^r}{r!} \frac{n^{-q}}{(r-q)!} = \\
    & \sum_{q=0}^{\infty} \left(\frac{s}{n}\right)^q \sum_{r=q}^{\infty} \frac{\left(zn\right)^r}{r!} \frac{1}{(r-q)!} \\
\end{align*}

Упростим внутреннюю сумму и введем функцию:
\begin{align*}
    & \sum_{r=q}^{\infty} \frac{\left(zn\right)^r}{r!} \frac{1}{(r-q)!} = \\
    & \sum_{r=0}^{\infty} \frac{\left(zn\right)^{r+q}}{(r+q)!} \frac{1}{r!} = \\
    & (zn)^q \sum_{r=0}^{\infty} \frac{\left(zn\right)^{r}}{r!(r+q)!} =: (zn)^q f_q(zn) \\
\end{align*}

Итак,
\begin{align*}
    & \sum_{q=0}^{\infty} \left(\frac{s}{n}\right)^q \sum_{r=q}^{\infty} \frac{\left(zn\right)^r}{r!} \frac{1}{(r-q)!} = \\
    & \sum_{q=0}^{\infty} \left(\frac{s}{n}\right)^q (zn)^q f_q(zn) = \\
    & \sum_{q=0}^{\infty} (sz)^q f_q(zn)
\end{align*}

Теперь вычислим искомый член для нужной нам функции:
\begin{align*}
    & e^{sz} e^{n / s} e^{-1 / 2s^2} = \\
    & \left[ \left(\sum_{q=-\infty}^{-1} s^q a_q(z)\right) + \left( \sum_{q=0}^{\infty} (sz)^q f_q(zn) \right)  \right] \left[ \sum_{q=0}^{\infty} \frac{-\left( \frac{1}{2s^2} \right)^q}{q!}  \right] = \\
    & \left(\sum_{q=-\infty}^{-1} b_q(z) s^q\right) + \sum_{q=0}^{\infty} s^q \sum_{r=0}^{\infty} \frac{\left( -\frac{1}{2} \right)^r}{r!} z^{2r+q} f_{2r+q}(zn)
\end{align*}

Отсюда, $(p-1)$-й член:
\begin{align*}
    & \sum_{r=0}^{\infty} \frac{\left( -\frac{1}{2} \right)^r}{r!} z^{2r+p-1} f_{2r+p-1}(zn) = \\
    & \sum_{r=0}^{\infty} \frac{\left( -\frac{1}{2} \right)^r}{r!} z^{2r+p-1} \sum_{t=0}^{\infty} \frac{\left(zn\right)^{t}}{t!(t+2r+p-1)!} = \\
    & \text{(что мы искали)} L^{-1}_s \left[ \frac{1}{s^p} \exp \left[ -\frac{1}{2 s^2} \right] \exp \left( \frac{n}{s} \right)  \right](z) \\
\end{align*}

Выразим теперь все три слагаемых:

\begin{align*}
    & L^{-1}_u \left[ \frac{1}{u^p} \exp \left[ -\frac{1}{2} \left( \frac{1}{u 2^{m+1}} \right)^2  \right] \exp \left( \frac{1}{u} \frac{n}{2^{m+1}}\right)   \right](z) = \\
    & 2^{p(m+1)} L^{-1}_u \left[ \frac{1}{\left( 2^{m+1}u \right)^p} \exp \left[ -\frac{1}{2} \left( \frac{1}{u 2^{m+1}} \right)^2  \right] \exp \left( \frac{1}{u} \frac{n}{2^{m+1}}\right)   \right](z) = \\
    & \text{(делаем замену $s = 2^{(m+1)}u$) } 2^{(p-1)(m+1)} L^{-1}_s \left[ \frac{1}{s^p} \exp \left[ -\frac{1}{2 s^2} \right] \exp \left( \frac{n}{s} \right)  \right] \left(\frac{z}{2^{m+1}}\right) = \\
    & 2^{(p-1)(m+1)} \sum_{r=0}^{\infty} \frac{\left( -\frac{1}{2} \right)^r}{r!} \left(\frac{z}{2^{m+1}}\right)^{2r+p-1} \sum_{t=0}^{\infty} \frac{\left(\frac{z}{2^{m+1}}n\right)^{t}}{t!(t+2r+p-1)!} \\
\end{align*}

Подставляем:

\begin{align*}
\begin{split}
    g(z) z^{k-1} \frac{1}{\Gamma(k)} \sqrt{2^m} \frac{\sqrt{3} \pi^{1 /4}}{2} e^{n^2 /2} ={}& (1-n^2) L^{-1}_u \left[ \frac{1}{u^k} \exp \left( -\frac{1}{2} \frac{1}{u^2} \frac{1}{4^{m+1}} \right) \exp \left( \frac{1}{u} \frac{n}{2^{m+1}} \right)  \right](z) \\
    +{}& \frac{n}{2^m} L^{-1}_u \left[ \frac{1}{u^{k+1}} \exp \left( -\frac{1}{2} \frac{1}{u^2} \frac{1}{4^{m+1}} \right) \exp \left( \frac{1}{u} \frac{n}{2^{m+1}} \right)   \right](z) \\ 
    -{}& \frac{1}{4^{m+1}} L^{-1}_u \left[ \frac{1}{u^{k+2}} \exp \left( -\frac{1}{2} \frac{1}{u^2} \frac{1}{4^{m+1}} \right) \exp \left( \frac{1}{u} \frac{n}{2^{m+1}} \right)   \right](z) \\
\end{split} \\
\begin{split}
    \implies g(z) z^{k-1} \frac{1}{\Gamma(k)} \sqrt{2^m} \frac{\sqrt{3} \pi^{1 /4}}{2} e^{n^2 /2} 
    ={}& (1-n^2) 2^{(k-1)(m+1)} \sum_{r=0}^{\infty} \frac{\left( -\frac{1}{2} \right)^r}{r!} \left(\frac{z}{2^{m+1}}\right)^{2r+k-1} \sum_{t=0}^{\infty} \frac{\left(\frac{z}{2^{m+1}}n\right)^{t}}{t!(t+2r+k-1)!} \\
    +{}& \frac{n}{2^m} 2^{k(m+1)} \sum_{r=0}^{\infty} \frac{\left( -\frac{1}{2} \right)^r}{r!} \left(\frac{z}{2^{m+1}}\right)^{2r+k} \sum_{t=0}^{\infty} \frac{\left(\frac{z}{2^{m+1}}n\right)^{t}}{t!(t+2r+k)!} \\
    -{}& \frac{1}{4^{m+1}} 2^{(k+1)(m+1)} \sum_{r=0}^{\infty} \frac{\left( -\frac{1}{2} \right)^r}{r!} \left(\frac{z}{2^{m+1}}\right)^{2r+k+1} \sum_{t=0}^{\infty} \frac{\left(\frac{z}{2^{m+1}}n\right)^{t}}{t!(t+2r+k+1)!} \\
\end{split}
\end{align*}

\section{Случай n=0}
Начнем решать.
\begin{align*}
    & \implies L\left[ g(z) z^{k-1} \right](u) = \left( \frac{1}{\sqrt{2} } \right)^m \Gamma(k) \frac{1}{u^k} \psi \left( \frac{1}{u} \frac{1}{2^{m+1}} \right) \\
    & \implies g(z) z^{k-1} = L^{-1}_u \left[ \frac{1}{\sqrt{2^m} } \Gamma(k) \frac{1}{u^k} \psi \left( \frac{1}{u} \frac{1}{2^{m+1}} \right)  \right] (z) \\
    & \implies g(z) z^{k-1} \frac{1}{\Gamma(k)} \sqrt{2^m} = L^{-1}_u \left[ \frac{1}{u^k} \psi \left( \frac{1}{u} \frac{1}{2^{m+1}} \right) \right] (z) \\
    & \implies g(z) z^{k-1} \frac{1}{\Gamma(k)} \sqrt{2^m} = L^{-1}_u \left[ \frac{1}{u^k} \frac{2}{\sqrt{3} \pi^{1 /4}} \left( 1 - \left( \frac{1}{u} \frac{1}{2^{m+1}} \right)^2  \right) \exp \left[ -\frac{1}{2} \left( \frac{1}{2^{m+1}} \frac{1}{u} \right)^2  \right]   \right](z) \\
\end{align*}

\begin{align*}
\begin{split}
    \implies g(z) z^{k-1} \frac{1}{\Gamma(k)} \sqrt{2^m} \frac{\sqrt{3} \pi^{1 /4}}{2}  =
        {}& L^{-1}_u \left[ \frac{1}{u^k} \exp \left( -\frac{1}{2} \frac{1}{u^2} \frac{1}{4^{m+1}} \right) \right](z) \\
       -{}& \frac{1}{4^{m+1}} L^{-1}_u \left[ \frac{1}{u^{k+2}} \exp \left( -\frac{1}{2} \frac{1}{u^2} \frac{1}{4^{m+1}} \right) \right](z) 
\end{split}
\end{align*}

Вычислим обратное преобразование:
\begin{align*}
    & L^{-1}_s \left[ \frac{1}{s^p} \exp \left[ -\frac{1}{2 s^2} \right]  \right](z) = \\
    & \frac{1}{2\pi i} \int_{\alpha - i \infty}^{\alpha+i\infty} e^{sz} \frac{1}{s^p} e^{-1/(2s^2)} ds = \\
\end{align*}
\begin{align*}
    & \frac{1}{2\pi i} \int_{\alpha - i \infty}^{\alpha+i\infty} e^{sz} \frac{1}{s^p} e^{-1/(2s^2)} ds = \\
    & \frac{1}{2\pi i} (2\pi i) \underset{0}{\Res} \left(e^{sz} \frac{1}{s^p} e^{-1/(2s^2)}  \right) = \\
    & c_{p-1} \text{,  --- $p-1$-й член ряда Лорана для } e^{sz} e^{-1/(2s^2)} \\
\end{align*}

Вычислим:
\begin{align*}
    & e^{sz} e^{-1 / (2s^2)} = \\
    & \left( \sum_{q=0}^{\infty} \frac{\left( sz \right)^q }{q!} \right)
      \left( \sum_{q=0}^{\infty} \frac{\left(-\frac{1}{2s^2}\right)^q}{q!} \right) 
\end{align*}

Отсюда $p-1$-й член ряда:
\begin{align*}
    & \sum_{r=0}^{\infty} \frac{z^{p-1+2r}}{\left( p-1+2r \right)!} \frac{\left( -\frac{1}{2} \right)^r}{r!} \\
\end{align*}

Подставляем:
\begin{align*}
\begin{split}
    \implies g(z) z^{k-1} \frac{1}{\Gamma(k)} \sqrt{2^m} \frac{\sqrt{3} \pi^{1 /4}}{2}  =
        {}& L^{-1}_u \left[ \frac{1}{u^k} \exp \left( -\frac{1}{2} \frac{1}{u^2} \frac{1}{4^{m+1}} \right) \right](z) \\
       -{}& \frac{1}{4^{m+1}} L^{-1}_u \left[ \frac{1}{u^{k+2}} \exp \left( -\frac{1}{2} \frac{1}{u^2} \frac{1}{4^{m+1}} \right) \right](z) 
\end{split} \\
\begin{split}
    \implies g(z) z^{k-1} \frac{1}{\Gamma(k)} \sqrt{2^m} \frac{\sqrt{3} \pi^{1 /4}}{2}  =
        {}& 2^{k(m+1)} L^{-1}_u \left[ \frac{1}{\left(u 2^{m+1}\right)^k} \exp \left( -\frac{1}{2} \left(\frac{1}{u} \frac{1}{2^{m+1}}\right)^2 \right) \right](z) \\
       -{}& 2^{(k+2)(m+1)} \frac{1}{4^{m+1}} L^{-1}_u \left[ \frac{1}{\left(u 2^{m+1}\right)^{k+2}} \exp \left( -\frac{1}{2} \left(\frac{1}{u} \frac{1}{2^{m+1}}\right)^2 \right) \right](z) 
\end{split} \\
\begin{split}
    \implies g(z) z^{k-1} \frac{1}{\Gamma(k)} \sqrt{2^m} \frac{\sqrt{3} \pi^{1 /4}}{2}  =
        {}& 2^{(k-1)(m+1)} L^{-1}_{s} \left[ \frac{1}{s^k} \exp \left( -\frac{1}{2s^2} \right) \right]\left(\frac{z}{2^{m+1}}\right) \\
       -{}& 2^{(k+1)(m+1)} \frac{1}{4^{m+1}} L^{-1}_s \left[ \frac{1}{s^{k+2}} \exp \left( -\frac{1}{2s^2}\right) \right]\left(\frac{z}{2^{m+1}}\right) 
\end{split} \\
\begin{split}
    \implies g(z) z^{k-1} \frac{1}{\Gamma(k)} \sqrt{2^m} \frac{\sqrt{3} \pi^{1 /4}}{2}  =
        {}& 2^{(k-1)(m+1)} \sum_{r=0}^{\infty} \frac{\left(\frac{z}{2^{m+1}}\right)^{k-1+2r}}{(k-1+2r)!} \frac{\left( -\frac{1}{2} \right)^r }{r!} \\
       -{}& 2^{(k+1)(m+1)} \frac{1}{4^{m+1}} \sum_{r=0}^{\infty} \frac{\left(\frac{z}{2^{m+1}}\right)^{k+2-1+2r}}{(k+2-1+2r)!} \frac{\left( -\frac{1}{2} \right)^r }{r!} 
\end{split}
\end{align*}

\section{Результат}
\begin{align*}
\begin{split}
     g_{m,0}(z) z^{k-1} \frac{1}{\Gamma(k)} \sqrt{2^m} \frac{\sqrt{3} \pi^{1 /4}}{2}  =
        {}& 2^{(k-1)(m+1)} \sum_{r=0}^{\infty} \frac{\left(\frac{z}{2^{m+1}}\right)^{k-1+2r}}{(k-1+2r)!} \frac{\left( -\frac{1}{2} \right)^r }{r!} \\
       -{}& 2^{(k+1)(m+1)} \frac{1}{4^{m+1}} \sum_{r=0}^{\infty} \frac{\left(\frac{z}{2^{m+1}}\right)^{k+1+2r}}{(k+1+2r)!} \frac{\left( -\frac{1}{2} \right)^r }{r!} \\
\end{split} \\
\begin{split}
     g_{m,n}(z) z^{k-1} \frac{1}{\Gamma(k)} \sqrt{2^m} \frac{\sqrt{3} \pi^{1 /4}}{2} e^{n^2 /2} 
    ={}& (1-n^2) 2^{(k-1)(m+1)} \sum_{r=0}^{\infty} \frac{\left( -\frac{1}{2} \right)^r}{r!} \left(\frac{z}{2^{m+1}}\right)^{2r+k-1} \sum_{t=0}^{\infty} \frac{\left(\frac{z}{2^{m+1}}n\right)^{t}}{t!(t+2r+k-1)!} \\
    +{}& \frac{n}{2^m} 2^{k(m+1)} \sum_{r=0}^{\infty} \frac{\left( -\frac{1}{2} \right)^r}{r!} \left(\frac{z}{2^{m+1}}\right)^{2r+k} \sum_{t=0}^{\infty} \frac{\left(\frac{z}{2^{m+1}}n\right)^{t}}{t!(t+2r+k)!} \\
    -{}& \frac{1}{4^{m+1}} 2^{(k+1)(m+1)} \sum_{r=0}^{\infty} \frac{\left( -\frac{1}{2} \right)^r}{r!} \left(\frac{z}{2^{m+1}}\right)^{2r+k+1} \sum_{t=0}^{\infty} \frac{\left(\frac{z}{2^{m+1}}n\right)^{t}}{t!(t+2r+k+1)!} \\
\end{split}
\end{align*}


\end{document}
