\RequirePackage[l2tabu,orthodox]{nag}
%
\documentclass[12pt, a4paper]{report}
%
\usepackage[utf8]{inputenc}
\usepackage[T1,T2A]{fontenc}
\usepackage[russian]{babel}
%
\usepackage[intlimits]{amsmath}  % Basic math structures
\usepackage{amssymb}  % Extended symbols
\usepackage{amsthm}  % Theorem environment
\usepackage{bm}
%
\usepackage[a4paper,margin=3cm,includefoot,heightrounded]{geometry}
\usepackage[colorlinks,unicode]{hyperref}
\usepackage{layout}
\usepackage{dsfont}
\usepackage{mathrsfs}
\usepackage{mathtools}
\usepackage{multirow}
\usepackage{tikz}
\usepackage{titlesec}
\usepackage{pgf}
\usepackage{pgfplots}
\usepackage{graphicx}
%
\usepackage{wrapfig}
%
\usepackage[backend=biber]{biblatex}
\usepackage{subfiles}
%
\addbibresource{bibliography.bib}
%
\graphicspath{ {./figs/} }
%
\pgfplotsset{compat=1.16}
\usetikzlibrary{babel}
\usetikzlibrary{arrows}
\usetikzlibrary{decorations}
\usetikzlibrary{decorations.markings}
\usetikzlibrary{patterns}
%
%
\titleformat{\chapter}[block]{\normalfont\huge\bfseries\centering}{\S \thechapter.}{20pt}{\Huge}
\newcommand*{\CC}{\mathbb{C}}
\newcommand*{\RR}{\mathds{R}}
\newcommand*{\ZZ}{\mathds{Z}}
\newcommand*{\NN}{\mathds{N}}
\newcommand*{\QQ}{\mathds{Q}}

\newcommand*{\EV}{\mathbf{E}}
\newcommand*{\Var}{\mathbf{D}}
\newcommand*{\PR}{\mathbf{P}}

\newcommand*{\D}{\bm{D}}
\newcommand*{\R}{\bm{R}}
\newcommand{\BraceThree}{%
	\multirow{3}{*}{$\left\{\begin{array}{@{}l@{}} \null \\ \null \\ \null\end{array}\right.$}%
}

\newcommand{\BraceTwo}{%
	\multirow{2}{*}{$\left\{\begin{array}{@{}l@{}} \null \\ \null\end{array}\right.$}%
}

\renewcommand\qedsymbol{$\blacksquare$}

\theoremstyle{plain}
\newtheorem{Th}{Теорема}
\newtheorem*{Th*}{Теорема}
\newtheorem{Lem}{Лемма}
\newtheorem*{Lem*}{Лемма}
\newtheorem*{St}{Утверждение}
\newtheorem*{Prop}{Свойства}

\theoremstyle{definition}
\newtheorem*{Def}{Определение}
\newtheorem*{Ex}{Пример}
\newtheorem*{Nam}{Обозначение}
\newtheorem*{Agr}{Договоренность}

\theoremstyle{remark}
\newtheorem*{Rem}{Замечание}
\newtheorem*{Probl}{Упражнение}

\renewcommand{\proofname}{Доказательство}
\addto\captionsrussian{
	\renewcommand{\proofname}{Доказательство}
}

\newenvironment{amatrix}[1]{%
	\left(\begin{array}{@{}*{#1}{c}|c@{}}
	}{%
	\end{array}\right)
}

\DeclareMathOperator*{\Res}{Res}
\DeclareMathOperator{\PE}{E}
\DeclareMathOperator{\PP}{P}

\renewcommand{\Re}{\operatorname{Re}}
\renewcommand{\Im}{\operatorname{Im}}

\DeclareMathOperator*{\argmin}{arg\,min}
%
\begin{document}
%
\thispagestyle{empty}
\sloppy
\begin{titlepage}
\begin{center}
Московский государственный университет имени~М.~В.~Ломоносова\\
механико-математический факультет\\
кафедра математической статистики и случайных процессов\\

\centering
\includegraphics[width=0.3\textwidth]{mechmath.jpg}

\vspace*{100pt} Курсовая работа\\студента 503 группы \\
Купрякова Василия Юрьевича
\\
\vspace{10pt} {\Large{\textbf{Непараметрическая вейвлет-оценка плотности мультипликативно зашумленных данных}}\\}

\vspace*{40pt}

\begin{flushright}
Научный руководитель:\\ с.н.с., к.ф.-м.н.\\  Шкляев Александр Викторович\\
\end{flushright}

\vspace*{\fill} Москва, 2021
\end{center}
\end{titlepage}
\clearpage
%
\tableofcontents
%
\subfile{chapters/introduction}
%
\subfile{chapters/inverse-laplace}
%
\subfile{chapters/mellin-solution}
%
\subfile{chapters/fredholm}
%
\subfile{chapters/evaluation}
%
\subfile{chapters/multilength}
%
\subfile{chapters/results}
%
\chapter {Сведение задачи к вычислению обратного преобразования Лапласа}
Есть случайные величины $X, Y, Z$. Мы не знаем распределение $X$, знаем распределение $Y$ и наблюдаем $Z$. Кроме того, известно, что $Z = XY$, и что все величины непрерывны. Нужно оценить распределение $X$.

Мы будем использовать вейвлет «Mexican hat», потому что он прост и непрерывен. Его формула:
\[
    \psi(t) = \frac{2}{\sqrt{3} \pi^{1 / 4}} (1-t^2)e^{-t^2 / 2}
.\]

Определим элементы фрейма: \\
\[
    \psi_{m,n}(t) = 
    \frac{1}{\sqrt{2^m}} \psi\left( \frac{t}{2^m}-n \right) =
    \frac{1}{\sqrt{2^m} }\frac{2}{\sqrt{3} \pi^{1 / 4}} \left(1-\left( \frac{t}{2^m} - n \right)^2 \right) e^{-\left( \frac{t}{2^m} - n \right)^2 / 2}
.\]

Рассмотрим случай $Y \sim \chi^2_{2k}$; $X > \delta > 0$, абсолютно непрерывен. Плотность $Y$:
\[
    \chi^2_{2k} \sim \frac{1}{2^k} \frac{1}{\Gamma(k)} x^{k-1} e^{-x / 2}
.\]


Будем строить функции $g_{m,n}$ такие, что $\PE g_{m,n}(Z) = \PE \psi_{m,n}(X)$. Заметим, что достаточно выполнения:
\[
    \forall x \in \Im X \quad \PE g_{m,n} (xY) = \psi_{m,n}(x) 
.\]

Заменим мат. ожидание преобразованием Лапласа и раскроем $\psi_{m,n}$:
\[
    \left( \frac{1}{2x} \right)^k \frac{1}{\Gamma(k)} L_z \left[g_{m,n}(z) z^{k-1}\right] \left( \frac{1}{2x} \right) = 
    \left( \frac{1}{\sqrt{2} } \right)^m \psi_{m,n} \left( \frac{x}{2^m} - n \right) 
.\]

Сделаем замену $u = \frac{1}{2x}$:
\[
    u^k \frac{1}{\Gamma(k)} L_z \left[g_{m,n}(z) z^{k-1}\right] \left( u \right) = 
    \left( \frac{1}{\sqrt{2} } \right)^m \psi_{m,n} \left( \frac{1}{2^{m+1} u} - n \right) 
.\]

Используя обратное преобразование Лапласа, найдем $g_{m,n}(t)$:
\begin{gather*}
    u^k \frac{1}{\Gamma(k)} L_z \left[g_{m,n}(z) z^{k-1}\right] \left( u \right) = 
    \left( \frac{1}{\sqrt{2} } \right)^m \psi_{m,n} \left( \frac{1}{2^{m+1} u} - n \right)
\\
    L_z \left[g_{m,n}(z) z^{k-1}\right] \left( u \right) = 
    \frac{\Gamma(k)}{u^k} \left( \frac{1}{\sqrt{2} } \right)^m \psi_{m,n} \left( \frac{1}{2^{m+1} u} - n \right)
\\
    g_{m,n}(t) t^{k-1} = 
    L^{-1}_u \left[ \frac{\Gamma(k)}{u^k} \left( \frac{1}{\sqrt{2} } \right)^m \psi_{m,n} \left( \frac{1}{2^{m+1} u} - n \right) \right] (t)
\\
    g_{m,n}(t) = 
    \frac{1}{t^{k-1}} \frac{\Gamma(k)}{\sqrt{2^m}} L^{-1}_u \left[ \frac{1}{u^k} \psi_{m,n} \left( \frac{1}{2^{m+1} u} - n \right) \right] (t)
.\end{gather*}

Таким образом мы получили выражение для $g_{m,n}(t)$. Далее мы выразим его через ряды, используя формулу Меллина и основную теорему о вычетах.

%
\printbibliography
\end{document}
