\documentclass[../../paper.tex]{subfiles}
\begin{document}
\subsection{Результат}
Выпишем обе полученные формулы вместе:
\[
    g_{m,0}(t) =
%
    \frac{\Gamma(k)}{\sqrt{2^m}}
    \frac{2}{\sqrt{3} \pi^{1/4}} \left(
%
    \sum_{j=0}^{\infty} \frac{\left(\frac{t}{2^{m+1}}\right)^{2j} / (-2)^j}{(2j+k-1)!\,j!}
-
    \left( \frac{t^2}{4^{m+1}} \right) 
    \sum_{j=0}^{\infty} \frac{\left(\frac{t}{2^{m+1}}\right)^{2j} / (-2)^j}{(2j+k+1)!\,j!}
%
    \right)
.\]
\begin{multline*}
    g_{m,n}(t)
=
    \frac{\Gamma(k)}{\sqrt{2^m}} e^{-\frac{n^2}{2}} \frac{2}{\sqrt{3} \pi^{1/4}} \left(
    \left( 1-n^2 \right) 
    \sum_{i=0}^{\infty} \sum_{j=0}^{\infty} \frac{\left( \frac{t}{2^{m+1}} \right) ^{2j+i} / (-2)^j}{(2j+k-1+i)!\,j!} \frac{n^i}{i!}
\right. +\\+
    \left( \frac{nt}{2^m} \right) 
    \sum_{i=0}^{\infty} \sum_{j=0}^{\infty} \frac{\left( \frac{t}{2^{m+1}} \right) ^{2j+i} / (-2)^j}{(2j+k+i)!\,j!} \frac{n^i}{i!}
-\\- \left.
    \left( \frac{t^2}{4^{m+1}} \right) 
    \sum_{i=0}^{\infty} \sum_{j=0}^{\infty} \frac{\left( \frac{t}{2^{m+1}} \right) ^{2j+i} / (-2)^j}{(2j+k+1+i)!\,j!} \frac{n^i}{i!}
    \right)
.\end{multline*}
%
К сожалению, такой способ не привел к успеху из-за непригодности для численных методов.
\end{document}
