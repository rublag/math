\documentclass[../paper.tex]{subfiles}
\begin{document}
\subsection{Сведение задачи к вычислению обратного преобразования Лапласа}
Рассмотрим только случай $X_i > \delta > 0$.

Мы будем использовать вейвлет «Mexican hat», потому что он прост и непрерывен. Его формула:
\[
    \psi(t) = \frac{2}{\sqrt{3} \pi^{1 / 4}} (1-t^2)e^{-t^2 / 2}
.\]

Определим элементы фрейма: \\
\[
    \psi_{m,n}(t) = 
    \frac{1}{\sqrt{2^m}} \psi\left( \frac{t}{2^m}-n \right) =
    \frac{1}{\sqrt{2^m} }\frac{2}{\sqrt{3} \pi^{1 / 4}} \left(1-\left( \frac{t}{2^m} - n \right)^2 \right) e^{-\left( \frac{t}{2^m} - n \right)^2 / 2}
.\]


Напишем плотность $Y_i$:
\[
	f_Y(y) = \frac{1}{2^k} \frac{1}{\Gamma(k)} y^{k-1} e^{-y / 2}
.\]

Раскроем $\PE g_{m,n}(xY)$:
\[
	\PE g_{m,n}(xY)
	= \int_0^\infty g_{m,n}(xy) f_Y(y) dy
	= \int_0^\infty g_{m,n}(xy) \frac{1}{2^k} \frac{1}{\Gamma(k)} y^{k-1} e^{-y / 2} dy
.\]
Произведем замену $z = xy$:
\begin{align*}
	\PE g_{m,n}(xY)
	&= \frac{1}{2^k}\frac{1}{\Gamma(k)} \int_0^\infty g_{m,n}(z) \frac{z^{k-1}}{x^{k-1}} e^{-z/(2x)} \frac{dz}{x}
	\\&= \left(\frac{1}{2x}\right)^k \frac{1}{\Gamma(k)} \int_0^\infty g_{m,n}(z) z^{k-1} \exp\left(-z \frac{1}{2x}\right) dz
.\end{align*}
%
Заменим интеграл преобразованием Лапласа:
\[
	\PE g_{m,n}(xY)
	= \left( \frac{1}{2x} \right)^k \frac{1}{\Gamma(k)} L_z \left[g_{m,n}(z) z^{k-1}\right] \left( \frac{1}{2x} \right)
.\]
%
Получаем функциональное уравнение:
\[
	\left( \frac{1}{2x} \right)^k \frac{1}{\Gamma(k)} L_z \left[g_{m,n}(z) z^{k-1}\right] \left( \frac{1}{2x} \right)
	= \psi_{m,n}(x)
	= \left( \frac{1}{\sqrt{2} } \right)^m \psi \left( \frac{x}{2^m} - n \right) 
\]
%
Сделаем замену $u = \frac{1}{2x}$:
\[
    u^k \frac{1}{\Gamma(k)} L_z \left[g_{m,n}(z) z^{k-1}\right] \left( u \right)
    = \left( \frac{1}{\sqrt{2} } \right)^m \psi_{m,n} \left( \frac{1}{2^{m+1} u} - n \right) 
.\]
%
Перенесем множители в правую часть:
\[
    L_z \left[g_{m,n}(z) z^{k-1}\right] \left( u \right)
    = \frac{\Gamma(k)}{u^k} \left( \frac{1}{\sqrt{2} } \right)^m \psi_{m,n} \left( \frac{1}{2^{m+1} u} - n \right)
.\]
%
Произведем обратное преобразование Лапласа:
\[
    g_{m,n}(z) z^{k-1}
    = L^{-1}_u \left[ \frac{\Gamma(k)}{u^k} \left( \frac{1}{\sqrt{2} } \right)^m \psi_{m,n} \left( \frac{1}{2^{m+1} u} - n \right) \right] (z)
.\]
%
Выразим $g_{m,n}(z)$:
\[
    g_{m,n}(z)
    = \frac{1}{z^{k-1}} \frac{\Gamma(k)}{\sqrt{2^m}} L^{-1}_u \left[ \frac{1}{u^k} \psi_{m,n} \left( \frac{1}{2^{m+1} u} - n \right) \right] (z)
.\]

Таким образом мы получили выражение для $g_{m,n}(z)$. Далее мы выразим его через ряды, используя формулу Меллина и основную теорему о вычетах.
\end{document}
