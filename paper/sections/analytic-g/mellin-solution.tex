\documentclass[../paper.tex]{subfiles}
\graphicspath{{\subfix{../figs/}}}
\begin{document}
\subsection{Вычисление обратного преобразования Лапласа с помощью формулы Меллина и основной теоремы о вычетах}

Мы выразили $g_{m,n}(t)$ как:
\begin{equation}\label{eq:gmn}
    g_{m,n}(t) = 
    \frac{1}{t^{k-1}} \frac{\Gamma(k)}{\sqrt{2^m}} L^{-1}_u \left[ \frac{1}{u^k} \psi_{m,n} \left( \frac{1}{2^{m+1} u} - n \right) \right] (t).
\end{equation}

Теперь нужно вычислить обратное преобразование Лапласа. Для этого мы будем использовать формулу Меллина:
\[
    L^{-1}_s \left[ F \left( s \right)  \right](t) = 
    \frac{1}{2\pi i} \int_{\alpha - i \infty}^{\alpha + i \infty} e^{ts} F(s) ds.
\]

Итак, нам нужно вычислить:
\[
    L^{-1}_u \left[ \frac{1}{u^k} \psi_{m,n} \left( \frac{1}{2^{m+1} u} - n \right) \right] (t).
\]

Подставим вместо $\psi$ формулу нашего вейвлета:
\begin{multline*}
    L^{-1}_u \left[ \frac{1}{u^k} \psi_{m,n} \left( \frac{1}{2^{m+1} u} - n \right) \right] (t) =
\\%
    L^{-1}_u \left[ \frac{1}{u^k} \frac{2}{\sqrt{3} \pi^{1/4} } 
    \left( 1 - \left( \frac{1}{2^{m+1} u} - n \right)^2 \right) 
    \exp \left( -\frac{1}{2} \left( \frac{1}{2^{m+1}u} - n \right)^2  \right)
    \right](t).
\end{multline*}

Распишем множитель перед экспонентой:
\begin{multline*}
    1 - \left( \frac{1}{2^{m+1} u } - n \right)^2 =
    1 - \left( \frac{1}{2^{2(m+1)} u^2} - 2 \frac{1}{2^{m+1} u } n + n^2 \right) =
\\%
    \left( 1 - n^2 \right) + \frac{1}{u} \left(\frac{n}{2^m}\right) - \frac{1}{u^2} \left(\frac{1}{4^{m+1}}\right)
.\end{multline*}

Введем обозначение:
\[
    r_{m,n}(u) = \frac{2}{\sqrt{3} \pi^{1/4}} \exp \left( -\frac{1}{2} \left( \frac{1}{2^{m+1}u} - n \right)  \right) 
.\]

Таким образом,
\begin{multline}\label{eq:invlap_gen}
    L^{-1}_u \left[ \frac{1}{u^k} \psi_{m,n} \left( \frac{1}{2^{m+1} u} - n \right) \right] (t)
=\\=
    \left( 1 - n^2 \right)  L^{-1}_u \left[ \frac{1}{u^k} r_{m,n}(u) \right](t) +
    \left( \frac{n}{2^m} \right)  L^{-1}_u \left[ \frac{1}{u^{k+1}} r_{m,n}(u) \right](t)
-\\-
    \left( \frac{1}{4^{m+1}} \right)  L^{-1}_u \left[ \frac{1}{u^{k+2}} r_{m,n}(u) \right](t)
.\end{multline}

Отсюда видно, что достаточно найти $L^{-1}_u [\frac{1}{u^k} r_{m,n}(u)](t)$ для каждого $k$.

\subsubsection{Находим $L^{-1}_u [\frac{1}{u^k} r_{m,n}(u)](t)$}
Выше мы ввели
\[
    L^{-1}_u \left[ \frac{1}{u^k} r_{m,n}(u) \right](t)
.\]
Подставим обратно $r_{m,n}(u)$:
\[
    L^{-1}_u \left[ \frac{1}{u^k} r_{m,n}(u) \right](t) =
%
    L^{-1}_u \left[ \frac{1}{u^k} \frac{2}{\sqrt{3} \pi^{1/4} } 
    \exp \left( -\frac{1}{2} \left( \frac{1}{2^{m+1}u} - n \right)^2  \right)
    \right](t)
.\]
Раскроем квадрат под экспонентой:
\begin{multline*}
    L^{-1}_u \left[ \frac{1}{u^k} r_{m,n}(u) \right](t)
=\\=%
    L^{-1}_u \left[ \frac{1}{u^k} \frac{2}{\sqrt{3} \pi^{1/4} } 
    \exp \left( -\frac{1}{2}
    \left(\left( n^2 \right) - \frac{1}{u} \left(\frac{n}{2^m}\right) + \frac{1}{u^2} \left(\frac{1}{4^{m+1}}\right)\right)
    \right)
    \right](t)
.\end{multline*}
Сгруппируем $2^{m+1}u$:
\begin{multline*}
    L^{-1}_u \left[ \frac{1}{u^k} r_{m,n}(u) \right](t)
=\\=%
    L^{-1}_u \left[ \frac{2^{k(m+1)}}{\left(2^{m+1}u\right)^k} \frac{2}{\sqrt{3} \pi^{1/4} } 
    \exp \left( -\frac{n^2}{2} + \frac{n}{2^{m+1}u} - \frac{1}{2 \left( 2^{m+1}u \right)^2 }
    \right)
    \right](t)
.\end{multline*}
Вынесем множители, не зависящие от $u$, за $L^{-1}_u$:
\begin{multline*}
    L^{-1}_u \left[ \frac{1}{u^k} r_{m,n}(u) \right](t)
=\\=
    e^{-\frac{n^2}{2}} 2^{k(m+1)} \frac{2}{\sqrt{3} \pi^{1/4}}
    L^{-1}_u \left[ \frac{1}{\left(2^{m+1}u\right)^k}
    \exp \left(\frac{n}{2^{m+1}u} - \frac{1}{2 \left( 2^{m+1}u \right)^2 }
    \right)
    \right](t)
.\end{multline*}
Используя теорему о замене переменной в обратном преобразовании Лапласа, делаем замену $s=2^{m+1}u$:
\[
    L^{-1}_u \left[ \frac{1}{u^k} r_{m,n}(u) \right](t) =
%
    e^{-\frac{n^2}{2}} 2^{(k-1)(m+1)} \frac{2}{\sqrt{3} \pi^{1/4}}
    L^{-1}_s \left[ \frac{1}{s^k}
    \exp \left(\frac{n}{s} - \frac{1}{2 s^2 }
    \right)
    \right] \left( \frac{t}{2^{m+1}} \right)
.\]
\subsubsection{Находим обратное преобразование Лапласа}
В предыдущем разделе мы выразили:
\begin{equation}\label{eq:invlap}
    L^{-1}_u \left[ \frac{1}{u^k} r_{m,n}(u) \right](t) =
%
    e^{-\frac{n^2}{2}} 2^{(k-1)(m+1)} \frac{2}{\sqrt{3} \pi^{1/4}}
    L^{-1}_s \left[ \frac{1}{s^k}
    \exp \left(\frac{n}{s} - \frac{1}{2 s^2 }
    \right)
    \right] \left( \frac{t}{2^{m+1}} \right)
.\end{equation}
Чтобы вычислить правую часть, найдем теперь
\begin{equation}\label{eq:invlap_as_int}
    L^{-1}_s \left[ \frac{1}{s^k}
    \exp \left(\frac{n}{s}\right)
    \exp \left(-\frac{1}{2 s^2 }\right)
    \right] \left( \tau \right)
.\end{equation}
Воспользуемся формулой Меллина обратного преобразования Лапласа. У наc особенность только в нуле, поэтому можно взять любую $\alpha>0$:
\[
    L^{-1}_s \left[ \frac{1}{s^k}
    \exp \left(\frac{n}{s}\right)
    \exp \left(-\frac{1}{2 s^2 }\right)
    \right] \left( \tau \right) =
%
    \frac{1}{2\pi i}\int_{\alpha-i\infty}^{\alpha+i\infty} e^{s\tau} \frac{1}{s^k} e^{-1/(2s^2)} e^{n/s} ds
.\]

\usetikzlibrary{arrows}
\usetikzlibrary{decorations.markings}
\usetikzlibrary{patterns}
%
%
\begin{tikzpicture}
    \begin{axis} [
        axis on top,
        axis lines=middle,
        xmin=-5,
        xmax=5,
        ymin=-5,
        ymax=5,
        unit vector ratio=1 1 1
        ]
%
%        \fill[pattern=north west lines, opacity=0.6] (-10,10) rectangle (2,-10);
%        \fill[white] (2, -3) -- (2, 3) arc(90:270:3) -- cycle;
%        \draw[dashed] (2, -10) -- (2, -3) arc(270:90:3) -- (2,10);
%
%
        \begin{scope}[decoration={
            markings,
            mark=between positions 0 and 1 step 5mm with {\arrow {stealth}}}
            ]
%
            \draw [postaction=decorate] (2,-4) -- (2,4) -- plot [domain=1:3, variable=\t] ({2 + 4*cos(\t * pi / 2 r)}, {4*sin(\t * pi / 2 r)}) -- cycle;
        \end{scope}
        \draw[fill=black] (0, 0) circle (2pt);
    \end{axis}
\end{tikzpicture}
%


Берем контур $C$ = $C_1 + C_2$, где $C_1$ --- искомый, а $C_2$ --- дуга окружности (слева от $C_1$ с центром в $(\alpha, 0)$).

Оценим $F(s) := (1/s^k) e^{-1/(2s^2)} e^{n/s}$ на $C_r$, где $r > 4\alpha$. Для этого оценим каждый из множителей. Сначала $1/s$:
\begin{multline*}
    \left|\frac{1}{s}\right| = 
    \left|\frac{1}{\alpha+re^{i\phi}}\right| =
    \frac{1}{\sqrt{\left(\alpha+r\cos\phi\right)^2 + \left(r\sin\phi\right)^2}} =
    \frac{1}{r\sqrt{\left(\frac{\alpha}{r} + \cos\phi\right)^2 + \sin^2\phi}}
\le \\ \le
    \frac{1}{r\sqrt{1 + \frac{2\alpha\cos\phi}{r}}} \le
    \frac{1}{r\sqrt{1 + \frac{\cos\phi}{2}}} \le
    \frac{\sqrt{2}}{r} 
.\end{multline*}
Теперь оценим $e^{-1/(2s^2)}$ на том же контуре. Известно, что $|e^{z}| = e^{|z|}$
\[
    \left|\exp\left(-\frac{1}{2s^2}\right)\right| \le
    \exp\left(\left|-\frac{1}{2s^2}\right|\right) =
    \exp\left(\frac{1}{r^2}\right)
.\]
Аналогично оцениваем $e^{n/s}$:
\[
    \left| \exp\left( \frac{n}{s} \right)  \right| \le 
    \exp\left(\left| \frac{n}{s} \right|\right) =
    \exp\left( \frac{|n|\sqrt{2} }{r} \right) 
.\]
Объединяем оценки и получаем:
\[
    \left|\frac{1}{s^k} \exp\left(-\frac{1}{2s^2}\right) \exp\left(\frac{n}{s}\right)\right| \le
    \left(\frac{\sqrt{2}}{r}\right)^k \exp\left(\frac{1}{r^2}\right) \exp\left(\frac{|n|\sqrt{2}}{r}\right)
    \xrightarrow[r \to \infty]{} 0
.\]

А значит, по лемме Жордана $\int\limits_{C_r} e^{s\tau} F(s)ds$ стремится к нулю.
Поэтому можем использовать основную теорему о вычетах:
\[
    \frac{1}{2\pi i}\int_{\alpha-i\infty}^{\alpha+i\infty} e^{s\tau} \frac{1}{s^k} e^{-1/(2s^2)} e^{n/s} ds =
    \frac{1}{2\pi i} 2\pi i \Res_0 \left( e^{s\tau} \frac{1}{s^k} e^{-1/(2s^2)} e^{n/s} ds \right)
.\]
У нас возникает два случая: $n=0$ и $n\neq$0
\subsubsection{Случай $n = 0$}
Воспользуемся теоремой о правильной части функции $e^{s\tau} e^{-1/(2s^2)}$. Нам нужен $k-1$-й член ряда Лорана. Получаем:
\[
    \frac{1}{2\pi i}\int_{\alpha-i\infty}^{\alpha+i\infty} e^{s\tau} \frac{1}{s^k} e^{-1/(2s^2)} e^{n/s} ds =
    \sum_{j=0}^{\infty} \frac{\tau^{2j+k-1} / (-2)^j}{(2j+k-1)!\,j!}
.\]
Таким образом, мы выразили вычислили \ref{eq:invlap_as_int}:
\[
    L^{-1}_s \left[ \frac{1}{s^k}
    \exp \left(\frac{n}{s}\right)
    \exp \left(-\frac{1}{2 s^2 }\right)
    \right] \left( \tau \right)
=
    \sum_{j=0}^{\infty} \frac{\tau^{2j+k-1} / (-2)^j}{(2j+k-1)!\,j!}
.\]
Подставим это выражение в \ref{eq:invlap}, заменяя $\tau$ на $t/2^{m+1}$:
\begin{multline*}
    L^{-1}_u \left[ \frac{1}{u^k} r_{m,n}(u) \right](t)
=\\=%
    e^{-\frac{n^2}{2}} 2^{(k-1)(m+1)} \frac{2}{\sqrt{3} \pi^{1/4}}
    L^{-1}_s \left[ \frac{1}{s^k}
    \exp \left(\frac{n}{s} - \frac{1}{2 s^2 }
    \right)
    \right] \left( \frac{t}{2^{m+1}} \right) =
\\%
    e^{-\frac{n^2}{2}} 2^{(k-1)(m+1)} \frac{2}{\sqrt{3} \pi^{1/4}}
    \sum_{j=0}^{\infty} \frac{\left(\frac{t}{2^{m+1}}\right)^{2j+k-1} / (-2)^j}{(2j+k-1)!\,j!}
.\end{multline*}
Наконец, подставим это в~\ref{eq:invlap_gen}:
\begin{multline}
    L^{-1}_u \left[ \frac{1}{u^k} \psi_{m,n} \left( \frac{1}{2^{m+1} u} - n \right) \right] (t)
=\\=
    \left( 1 - n^2 \right)  L^{-1}_u \left[ \frac{1}{u^k} r_{m,n}(u) \right](t) +
    \left( \frac{n}{2^m} \right)  L^{-1}_u \left[ \frac{1}{u^{k+1}} r_{m,n}(u) \right](t)
+\\+
    \left( \frac{1}{4^{m+1}} \right)  L^{-1}_u \left[ \frac{1}{u^{k+2}} r_{m,n}(u) \right](t)
=\\=
    \left( 1-n^2 \right) 
    e^{-\frac{n^2}{2}} 2^{(k-1)(m+1)} \frac{2}{\sqrt{3} \pi^{1/4}}
    \sum_{j=0}^{\infty} \frac{\left(\frac{t}{2^{m+1}}\right)^{2j+k-1} / (-2)^j}{(2j+k-1)!\,j!}
+\\+
    \left( \frac{n}{2^m} \right) 
    e^{-\frac{n^2}{2}} 2^{k(m+1)} \frac{2}{\sqrt{3} \pi^{1/4}}
    \sum_{j=0}^{\infty} \frac{\left(\frac{t}{2^{m+1}}\right)^{2j+k} / (-2)^j}{(2j+k)!\,j!}
-\\-
    \left( \frac{1}{4^{m+1}} \right) 
    e^{-\frac{n^2}{2}} 2^{(k+1)(m+1)} \frac{2}{\sqrt{3} \pi^{1/4}}
    \sum_{j=0}^{\infty} \frac{\left(\frac{t}{2^{m+1}}\right)^{2j+k+1} / (-2)^j}{(2j+k+1)!\,j!}
.\end{multline}
Теперь получим выражение для $g(t)$, подставляя только что полученную формулу в~\ref{eq:gmn}:
\begin{multline*}
    g_{m,n}(t) = 
    \frac{1}{t^{k-1}} \frac{\Gamma(k)}{\sqrt{2^m}} L^{-1}_u \left[ \frac{1}{u^k} \psi_{m,n} \left( \frac{1}{2^{m+1} u} - n \right) \right] (t)
=\\=
    \frac{1}{t^{k-1}} \frac{\Gamma(k)}{\sqrt{2^m}} \left( 1-n^2 \right) 
    e^{-\frac{n^2}{2}} 2^{(k-1)(m+1)} \frac{2}{\sqrt{3} \pi^{1/4}}
    \sum_{j=0}^{\infty} \frac{\left(\frac{t}{2^{m+1}}\right)^{2j+k-1} / (-2)^j}{(2j+k-1)!\,j!}
+\\+
    \frac{1}{t^{k-1}} \frac{\Gamma(k)}{\sqrt{2^m}} \left( \frac{n}{2^m} \right) 
    e^{-\frac{n^2}{2}} 2^{k(m+1)} \frac{2}{\sqrt{3} \pi^{1/4}}
    \sum_{j=0}^{\infty} \frac{\left(\frac{t}{2^{m+1}}\right)^{2j+k} / (-2)^j}{(2j+k)!\,j!}
-\\-
    \frac{1}{t^{k-1}} \frac{\Gamma(k)}{\sqrt{2^m}} \left( \frac{1}{4^{m+1}} \right) 
    e^{-\frac{n^2}{2}} 2^{(k+1)(m+1)} \frac{2}{\sqrt{3} \pi^{1/4}}
    \sum_{j=0}^{\infty} \frac{\left(\frac{t}{2^{m+1}}\right)^{2j+k+1} / (-2)^j}{(2j+k+1)!\,j!}
.\end{multline*}
%
Начнем упрощать выражение. Для начала, вынесем из суммы, степень, не зависящую от переменной суммирования и сделаем замену $n=0$ (так как рассматриваем именно этот случай):
\[
    g_{m,0}(t) =
%
    \frac{\Gamma(k)}{\sqrt{2^m}}
    \frac{2}{\sqrt{3} \pi^{1/4}} \left(
%
    \sum_{j=0}^{\infty} \frac{\left(\frac{t}{2^{m+1}}\right)^{2j} / (-2)^j}{(2j+k-1)!\,j!}
-
    \left( \frac{t^2}{4^{m+1}} \right) 
    \sum_{j=0}^{\infty} \frac{\left(\frac{t}{2^{m+1}}\right)^{2j} / (-2)^j}{(2j+k+1)!\,j!}
%
    \right)
.\]
\subsubsection{Случай $n \neq 0$}
Воспользуемся теоремой о правильной части функции $e^{s\tau} e^{-1/(2s^2)} e^{n/s}$. Нам нужен $k-1$-й член ряда Лорана. Получаем:
\[
    \frac{1}{2\pi i}\int_{\alpha-i\infty}^{\alpha+i\infty} e^{s\tau} \frac{1}{s^k} e^{-1/{2s^2}} e^{n/s} ds =
    \sum_{i=0}^{\infty} \sum_{j=0}^{\infty} \frac{\tau^{2j+k-1+i} / (-2)^j}{(2j+k-1+i)!\,j!} \frac{n^i}{i!}
.\]
Таким образом, мы выразили вычислили \ref{eq:invlap_as_int}:
\[
    L^{-1}_s \left[ \frac{1}{s^k}
    \exp \left(\frac{n}{s}\right)
    \exp \left(-\frac{1}{2 s^2 }\right)
    \right] \left( \tau \right)
=
    \sum_{i=0}^{\infty} \sum_{j=0}^{\infty} \frac{\tau^{2j+k-1+i} / (-2)^j}{(2j+k-1+i)!\,j!} \frac{n^i}{i!}
.\]
Подставим это выражение в \ref{eq:invlap}, заменяя $\tau$ на $t/2^{m+1}$:
\begin{multline*}
    L^{-1}_u \left[ \frac{1}{u^k} r_{m,n}(u) \right](t)
=\\=%
    e^{-\frac{n^2}{2}} 2^{(k-1)(m+1)} \frac{2}{\sqrt{3} \pi^{1/4}}
    L^{-1}_s \left[ \frac{1}{s^k}
    \exp \left(\frac{n}{s} - \frac{1}{2 s^2 }
    \right)
    \right] \left( \frac{t}{2^{m+1}} \right)
=\\=%
    e^{-\frac{n^2}{2}} 2^{(k-1)(m+1)} \frac{2}{\sqrt{3} \pi^{1/4}}
    \sum_{i=0}^{\infty} \sum_{j=0}^{\infty} \frac{\left( \frac{t}{2^{m+1}} \right) ^{2j+k-1+i} / (-2)^j}{(2j+k-1+i)!\,j!} \frac{n^i}{i!}
.\end{multline*}
Наконец, подставим это в~\ref{eq:invlap_gen}:
\begin{multline}
    L^{-1}_u \left[ \frac{1}{u^k} \psi_{m,n} \left( \frac{1}{2^{m+1} u} - n \right) \right] (t) 
=\\=
    \left( 1 - n^2 \right)  L^{-1}_u \left[ \frac{1}{u^k} r_{m,n}(u) \right](t) +
    \left( \frac{n}{2^m} \right)  L^{-1}_u \left[ \frac{1}{u^{k+1}} r_{m,n}(u) \right](t)
+\\+
    \left( \frac{1}{4^{m+1}} \right)  L^{-1}_u \left[ \frac{1}{u^{k+2}} r_{m,n}(u) \right](t)
=\\=
    \left( 1-n^2 \right) 
    e^{-\frac{n^2}{2}} 2^{(k-1)(m+1)} \frac{2}{\sqrt{3} \pi^{1/4}}
    \sum_{i=0}^{\infty} \sum_{j=0}^{\infty} \frac{\left( \frac{t}{2^{m+1}} \right) ^{2j+k-1+i} / (-2)^j}{(2j+k-1+i)!\,j!} \frac{n^i}{i!}
+\\+
    \left( \frac{n}{2^m} \right) 
    e^{-\frac{n^2}{2}} 2^{k(m+1)} \frac{2}{\sqrt{3} \pi^{1/4}}
    \sum_{i=0}^{\infty} \sum_{j=0}^{\infty} \frac{\left( \frac{t}{2^{m+1}} \right) ^{2j+k+i} / (-2)^j}{(2j+k+i)!\,j!} \frac{n^i}{i!}
-\\-
    \left( \frac{1}{4^{m+1}} \right) 
    e^{-\frac{n^2}{2}} 2^{(k+1)(m+1)} \frac{2}{\sqrt{3} \pi^{1/4}}
    \sum_{i=0}^{\infty} \sum_{j=0}^{\infty} \frac{\left( \frac{t}{2^{m+1}} \right) ^{2j+k+1+i} / (-2)^j}{(2j+k+1+i)!\,j!} \frac{n^i}{i!}
.\end{multline}
Теперь получим выражение для $g_{m,n}(t)$, подставляя только что полученную формулу в~\ref{eq:gmn}
\begin{multline*}
    g_{m,n}(t) = 
    \frac{1}{t^{k-1}} \frac{\Gamma(k)}{\sqrt{2^m}} L^{-1}_u \left[ \frac{1}{u^k} \psi_{m,n} \left( \frac{1}{2^{m+1} u} - n \right) \right] (t)
=\\=
    \frac{1}{t^{k-1}} \frac{\Gamma(k)}{\sqrt{2^m}} \left( 1-n^2 \right) 
    e^{-\frac{n^2}{2}} 2^{(k-1)(m+1)} \frac{2}{\sqrt{3} \pi^{1/4}}
    \sum_{i=0}^{\infty} \sum_{j=0}^{\infty} \frac{\left( \frac{t}{2^{m+1}} \right) ^{2j+k-1+i} / (-2)^j}{(2j+k-1+i)!\,j!} \frac{n^i}{i!}
+\\+
    \frac{1}{t^{k-1}} \frac{\Gamma(k)}{\sqrt{2^m}} \left( \frac{n}{2^m} \right) 
    e^{-\frac{n^2}{2}} 2^{k(m+1)} \frac{2}{\sqrt{3} \pi^{1/4}}
    \sum_{i=0}^{\infty} \sum_{j=0}^{\infty} \frac{\left( \frac{t}{2^{m+1}} \right) ^{2j+k+i} / (-2)^j}{(2j+k+i)!\,j!} \frac{n^i}{i!}
-\\-
    \frac{1}{t^{k-1}} \frac{\Gamma(k)}{\sqrt{2^m}} \left( \frac{1}{4^{m+1}} \right) 
    e^{-\frac{n^2}{2}} 2^{(k+1)(m+1)} \frac{2}{\sqrt{3} \pi^{1/4}}
    \sum_{i=0}^{\infty} \sum_{j=0}^{\infty} \frac{\left( \frac{t}{2^{m+1}} \right) ^{2j+k+1+i} / (-2)^j}{(2j+k+1+i)!\,j!} \frac{n^i}{i!}
.\end{multline*}
Упростим:
\begin{multline*}
    g_{m,n}(t)
=
    \frac{\Gamma(k)}{\sqrt{2^m}} e^{-\frac{n^2}{2}} \frac{2}{\sqrt{3} \pi^{1/4}} \left(
    \left( 1-n^2 \right) 
    \sum_{i=0}^{\infty} \sum_{j=0}^{\infty} \frac{\left( \frac{t}{2^{m+1}} \right) ^{2j+i} / (-2)^j}{(2j+k-1+i)!\,j!} \frac{n^i}{i!} \right.
+\\+
    \left( \frac{nt}{2^m} \right) 
    \sum_{i=0}^{\infty} \sum_{j=0}^{\infty} \frac{\left( \frac{t}{2^{m+1}} \right) ^{2j+i} / (-2)^j}{(2j+k+i)!\,j!} \frac{n^i}{i!}
-\\- \left.
    \left( \frac{t^2}{4^{m+1}} \right) 
    \sum_{i=0}^{\infty} \sum_{j=0}^{\infty} \frac{\left( \frac{t}{2^{m+1}} \right) ^{2j+i} / (-2)^j}{(2j+k+1+i)!\,j!} \frac{n^i}{i!}
    \right)
.\end{multline*}
\subsection{Результат}
Выпишем обе полученные формулы вместе:
\[
    g_{m,0}(t) =
%
    \frac{\Gamma(k)}{\sqrt{2^m}}
    \frac{2}{\sqrt{3} \pi^{1/4}} \left(
%
    \sum_{j=0}^{\infty} \frac{\left(\frac{t}{2^{m+1}}\right)^{2j} / (-2)^j}{(2j+k-1)!\,j!}
-
    \left( \frac{t^2}{4^{m+1}} \right) 
    \sum_{j=0}^{\infty} \frac{\left(\frac{t}{2^{m+1}}\right)^{2j} / (-2)^j}{(2j+k+1)!\,j!}
%
    \right)
.\]
\begin{multline*}
    g_{m,n}(t)
=
    \frac{\Gamma(k)}{\sqrt{2^m}} e^{-\frac{n^2}{2}} \frac{2}{\sqrt{3} \pi^{1/4}} \left(
    \left( 1-n^2 \right) 
    \sum_{i=0}^{\infty} \sum_{j=0}^{\infty} \frac{\left( \frac{t}{2^{m+1}} \right) ^{2j+i} / (-2)^j}{(2j+k-1+i)!\,j!} \frac{n^i}{i!}
\right. +\\+
    \left( \frac{nt}{2^m} \right) 
    \sum_{i=0}^{\infty} \sum_{j=0}^{\infty} \frac{\left( \frac{t}{2^{m+1}} \right) ^{2j+i} / (-2)^j}{(2j+k+i)!\,j!} \frac{n^i}{i!}
-\\- \left.
    \left( \frac{t^2}{4^{m+1}} \right) 
    \sum_{i=0}^{\infty} \sum_{j=0}^{\infty} \frac{\left( \frac{t}{2^{m+1}} \right) ^{2j+i} / (-2)^j}{(2j+k+1+i)!\,j!} \frac{n^i}{i!}
    \right)
.\end{multline*}
%
К сожалению, такой способ не привел к успеху из-за непригодности для численных методов.
\end{document}
