\documentclass[../paper.tex]{subfiles}
\graphicspath{{\subfix{../figs/}}}
\begin{document}
\subsection{Вычисление обратного преобразование Лапласа}
В предыдущем разделе было получено соотношение:
\begin{equation}\label{eq:invlap}
    L^{-1}_u \left[ \frac{1}{u^k} r_{m,n}(u) \right](t) =
%
    e^{-\frac{n^2}{2}} 2^{(k-1)(m+1)} \frac{2}{\sqrt{3} \pi^{1/4}}
    L^{-1}_s \left[ \frac{1}{s^k}
    \exp \left(\frac{n}{s} - \frac{1}{2 s^2 }
    \right)
    \right] \left( \frac{t}{2^{m+1}} \right)
.\end{equation}
Чтобы вычислить правую часть, найдем величину
\begin{equation}\label{eq:invlap_as_int}
    L^{-1}_s \left[ \frac{1}{s^k}
    \exp \left(\frac{n}{s}\right)
    \exp \left(-\frac{1}{2 s^2 }\right)
    \right] \left( \tau \right)
.\end{equation}
Воспользуемся формулой Меллина обратного преобразования Лапласа. Данная функция имеет особенность только в нуле, откуда $\alpha$ можеть быть выбрано произвольным положительным числом.
Отсюда
\[
    L^{-1}_s \left[ \frac{1}{s^k}
    \exp \left(\frac{n}{s}\right)
    \exp \left(-\frac{1}{2 s^2 }\right)
    \right] \left( \tau \right) =
%
    \frac{1}{2\pi i}\int_{\alpha-i\infty}^{\alpha+i\infty} e^{s\tau} \frac{1}{s^k} e^{-1/(2s^2)} e^{n/s} ds
.\]
Рассмотрим контур $C$ = $C_1 + C_2$, где $C_1$ --- искомый, а $C_2$ --- дуга окружности (слева от $C_1$ с центром в $(\alpha, 0)$).
\begin{figure}[ht]
	\centering
	\usetikzlibrary{arrows}
\usetikzlibrary{decorations.markings}
\usetikzlibrary{patterns}
%
%
\begin{tikzpicture}
    \begin{axis} [
        axis on top,
        axis lines=middle,
        xmin=-5,
        xmax=5,
        ymin=-5,
        ymax=5,
        unit vector ratio=1 1 1
        ]
%
%        \fill[pattern=north west lines, opacity=0.6] (-10,10) rectangle (2,-10);
%        \fill[white] (2, -3) -- (2, 3) arc(90:270:3) -- cycle;
%        \draw[dashed] (2, -10) -- (2, -3) arc(270:90:3) -- (2,10);
%
%
        \begin{scope}[decoration={
            markings,
            mark=between positions 0 and 1 step 5mm with {\arrow {stealth}}}
            ]
%
            \draw [postaction=decorate] (2,-4) -- (2,4) -- plot [domain=1:3, variable=\t] ({2 + 4*cos(\t * pi / 2 r)}, {4*sin(\t * pi / 2 r)}) -- cycle;
        \end{scope}
        \draw[fill=black] (0, 0) circle (2pt);
    \end{axis}
\end{tikzpicture}
%

	\caption{Наглядное изображение контура.}
\end{figure}

Оценим $F(s) := (1/s^k) e^{-1/(2s^2)} e^{n/s}$ на $C_r$ при $r > 4\alpha$. Для этого оценим каждый из множителей. Заметим, что
\begin{multline*}
    \left|\frac{1}{s}\right| = 
    \left|\frac{1}{\alpha+re^{i\phi}}\right| =
    \frac{1}{\sqrt{\left(\alpha+r\cos\phi\right)^2 + \left(r\sin\phi\right)^2}} =
    \frac{1}{r\sqrt{\left(\frac{\alpha}{r} + \cos\phi\right)^2 + \sin^2\phi}}
\le \\ \le
    \frac{1}{r\sqrt{1 + \frac{2\alpha\cos\phi}{r}}} \le
    \frac{1}{r\sqrt{1 + \frac{\cos\phi}{2}}} \le
    \frac{\sqrt{2}}{r} 
.\end{multline*}
Теперь оценим $e^{-1/(2s^2)}$ на том же контуре. Известно, что $|e^{z}| \le e^{|z|}$
\[
    \left|\exp\left(-\frac{1}{2s^2}\right)\right| \le
    \exp\left(\left|-\frac{1}{2s^2}\right|\right) =
    \exp\left(\frac{1}{r^2}\right)
.\]
Аналогично оцениваем $e^{n/s}$:
\[
    \left| \exp\left( \frac{n}{s} \right)  \right| \le 
    \exp\left(\left| \frac{n}{s} \right|\right) =
    \exp\left( \frac{|n|\sqrt{2} }{r} \right) 
.\]
Объединяя полученные оценки, получаем
\[
    \left|\frac{1}{s^k} \exp\left(-\frac{1}{2s^2}\right) \exp\left(\frac{n}{s}\right)\right| \le
    \left(\frac{\sqrt{2}}{r}\right)^k \exp\left(\frac{1}{r^2}\right) \exp\left(\frac{|n|\sqrt{2}}{r}\right)
    \xrightarrow[r \to \infty]{} 0
.\]

Применяя модифицированную лемму Жордана (\ref{modified-jordan}), получаем, что интеграл $\int\limits_{C_r} e^{s\tau} F(s)ds$ стремится к нулю при $r \to \infty$.
Следовательно, в силу теоремы Коши о вычетах:
\[
    \frac{1}{2\pi i}\int_{\alpha-i\infty}^{\alpha+i\infty} e^{s\tau} \frac{1}{s^k} e^{-1/(2s^2)} e^{n/s} ds =
    \frac{1}{2\pi i} 2\pi i \Res_0 \left( e^{s\tau} \frac{1}{s^k} e^{-1/(2s^2)} e^{n/s} ds \right)
.\]
Здесь возникает два различных случая: $n=0$ и $n\neq$0

\subsubsection{Случай $n = 0$}
Нам нужен $k-1$--й член ряда Лорана функции $e^{s\tau} e^{-1/(2s^2)}$. Воспользуемся леммой \ref{laurent-2} о правильной части этой функции.  Получаем:
\[
    \frac{1}{2\pi i}\int_{\alpha-i\infty}^{\alpha+i\infty} e^{s\tau} \frac{1}{s^k} e^{-1/(2s^2)} e^{n/s} ds =
    \sum_{j=0}^{\infty} \frac{\tau^{2j+k-1} (-2)^{-j}}{(2j+k-1)!\,j!}
.\]
Таким образом, мы вычислили выражение (\ref{eq:invlap_as_int}) для $n = 0$:
\[
	L^{-1}_s \left[ \frac{1}{s^k}
		\exp \left(\frac{n}{s}\right)
		\exp \left(-\frac{1}{2 s^2 }\right)
	\right] \left( \tau \right)
	=
	\sum_{j=0}^{\infty} \frac{\tau^{2j+k-1} (-2)^{-j}}{(2j+k-1)!\,j!}
.\]
Подставим этот ряд в выражение (\ref{eq:invlap}), заменяя $\tau$ на $t/2^{m+1}$:
\begin{multline*}
    L^{-1}_u \left[ \frac{1}{u^k} r_{m,n}(u) \right](t)
=\\=%
    e^{-\frac{n^2}{2}} 2^{(k-1)(m+1)} \frac{2}{\sqrt{3} \pi^{1/4}}
    L^{-1}_s \left[ \frac{1}{s^k}
    \exp \left(\frac{n}{s} - \frac{1}{2 s^2 }
    \right)
    \right] \left( \frac{t}{2^{m+1}} \right) =
\\%
    e^{-\frac{n^2}{2}} 2^{(k-1)(m+1)} \frac{2}{\sqrt{3} \pi^{1/4}}
    \sum_{j=0}^{\infty} \frac{\left(\frac{t}{2^{m+1}}\right)^{2j+k-1} (-2)^{-j}}{(2j+k-1)!\,j!}
.\end{multline*}
Наконец, подставим получившееся выражение в формулу~(\ref{eq:invlap_gen}):
\begin{multline}\label{local:0:rmn}
    L^{-1}_u \left[ \frac{1}{u^k} \psi_{m,n} \left( \frac{1}{2^{m+1} u} - n \right) \right] (t)
=\\=
    \left( 1 - n^2 \right)  L^{-1}_u \left[ \frac{1}{u^k} r_{m,n}(u) \right](t) +
    \left( \frac{n}{2^m} \right)  L^{-1}_u \left[ \frac{1}{u^{k+1}} r_{m,n}(u) \right](t)
+\\+
    \left( \frac{1}{4^{m+1}} \right)  L^{-1}_u \left[ \frac{1}{u^{k+2}} r_{m,n}(u) \right](t)
=\\=
    \left( 1-n^2 \right) 
    e^{-\frac{n^2}{2}} 2^{(k-1)(m+1)} \frac{2}{\sqrt{3} \pi^{1/4}}
    \sum_{j=0}^{\infty} \frac{\left(\frac{t}{2^{m+1}}\right)^{2j+k-1} (-2)^{-j}}{(2j+k-1)!\,j!}
+\\+
    \left( \frac{n}{2^m} \right) 
    e^{-\frac{n^2}{2}} 2^{k(m+1)} \frac{2}{\sqrt{3} \pi^{1/4}}
    \sum_{j=0}^{\infty} \frac{\left(\frac{t}{2^{m+1}}\right)^{2j+k} (-2)^{-j}}{(2j+k)!\,j!}
-\\-
    \left( \frac{1}{4^{m+1}} \right) 
    e^{-\frac{n^2}{2}} 2^{(k+1)(m+1)} \frac{2}{\sqrt{3} \pi^{1/4}}
    \sum_{j=0}^{\infty} \frac{\left(\frac{t}{2^{m+1}}\right)^{2j+k+1} (-2)^{-j}}{(2j+k+1)!\,j!}
.\end{multline}
Теперь получим выражение для $g_{m,n}(t)$, подставляя соотношение (\ref{local:0:rmn}) в выражение~(\ref{eq:gmn-laplace}):
\begin{multline*}
    g_{m,n}(t) = 
    \frac{1}{t^{k-1}} \frac{\Gamma(k)}{\sqrt{2^m}} L^{-1}_u \left[ \frac{1}{u^k} \psi_{m,n} \left( \frac{1}{2^{m+1} u} - n \right) \right] (t)
=\\=
    \frac{1}{t^{k-1}} \frac{\Gamma(k)}{\sqrt{2^m}} \left( 1-n^2 \right) 
    e^{-\frac{n^2}{2}} 2^{(k-1)(m+1)} \frac{2}{\sqrt{3} \pi^{1/4}}
    \sum_{j=0}^{\infty} \frac{\left(\frac{t}{2^{m+1}}\right)^{2j+k-1} (-2)^{-j}}{(2j+k-1)!\,j!}
+\\+
    \frac{1}{t^{k-1}} \frac{\Gamma(k)}{\sqrt{2^m}} \left( \frac{n}{2^m} \right) 
    e^{-\frac{n^2}{2}} 2^{k(m+1)} \frac{2}{\sqrt{3} \pi^{1/4}}
    \sum_{j=0}^{\infty} \frac{\left(\frac{t}{2^{m+1}}\right)^{2j+k} (-2)^{-j}}{(2j+k)!\,j!}
-\\-
    \frac{1}{t^{k-1}} \frac{\Gamma(k)}{\sqrt{2^m}} \left( \frac{1}{4^{m+1}} \right) 
    e^{-\frac{n^2}{2}} 2^{(k+1)(m+1)} \frac{2}{\sqrt{3} \pi^{1/4}}
    \sum_{j=0}^{\infty} \frac{\left(\frac{t}{2^{m+1}}\right)^{2j+k+1} (-2)^{-j}}{(2j+k+1)!\,j!}
.\end{multline*}
%
Теперь упростим выражение. Вынесем из суммы, степень, не зависящую от переменной суммирования и сделаем замену $n=0$ (так как рассматриваем именно этот случай):
\[
    g_{m,0}(t) =
%
    \frac{\Gamma(k)}{\sqrt{2^m}}
    \frac{2}{\sqrt{3} \pi^{1/4}} \left(
%
    \sum_{j=0}^{\infty} \frac{\left(\frac{t}{2^{m+1}}\right)^{2j} (-2)^{-j}}{(2j+k-1)!\,j!}
-
    \left( \frac{t^2}{4^{m+1}} \right) 
    \sum_{j=0}^{\infty} \frac{\left(\frac{t}{2^{m+1}}\right)^{2j} (-2)^{-j}}{(2j+k+1)!\,j!}
%
    \right)
.\]
\subsubsection{Случай $n \neq 0$}
Нам нужен $k-1$-й член ряда Лорана функции $e^{s\tau} e^{-1/(2s^2)} e^{n/s}$. Воспользуемся леммой \ref{laurent-3} о правильной части этой функции. Получаем:
\[
    \frac{1}{2\pi i}\int_{\alpha-i\infty}^{\alpha+i\infty} e^{s\tau} \frac{1}{s^k} e^{-1/{2s^2}} e^{n/s} ds =
    \sum_{i=0}^{\infty} \sum_{j=0}^{\infty} \frac{\tau^{2j+k-1+i} (-2)^{-j}}{(2j+k-1+i)!\,j!} \frac{n^i}{i!}
.\]
Таким образом, мы вычислили выражение (\ref{eq:invlap_as_int}) для $n \ne 0$:
\[
    L^{-1}_s \left[ \frac{1}{s^k}
    \exp \left(\frac{n}{s}\right)
    \exp \left(-\frac{1}{2 s^2 }\right)
    \right] \left( \tau \right)
=
    \sum_{i=0}^{\infty} \sum_{j=0}^{\infty} \frac{\tau^{2j+k-1+i} (-2)^{-j}}{(2j+k-1+i)!\,j!} \frac{n^i}{i!}
.\]
Подставим это выражение в выражение (\ref{eq:invlap}), заменяя $\tau$ на $t/2^{m+1}$:
\begin{multline*}
    L^{-1}_u \left[ \frac{1}{u^k} r_{m,n}(u) \right](t)
=\\=%
    e^{-\frac{n^2}{2}} 2^{(k-1)(m+1)} \frac{2}{\sqrt{3} \pi^{1/4}}
    L^{-1}_s \left[ \frac{1}{s^k}
    \exp \left(\frac{n}{s} - \frac{1}{2 s^2 }
    \right)
    \right] \left( \frac{t}{2^{m+1}} \right)
=\\=%
    e^{-\frac{n^2}{2}} 2^{(k-1)(m+1)} \frac{2}{\sqrt{3} \pi^{1/4}}
    \sum_{i=0}^{\infty} \sum_{j=0}^{\infty} \frac{\left( \frac{t}{2^{m+1}} \right) ^{2j+k-1+i} (-2)^{-j}}{(2j+k-1+i)!\,j!} \frac{n^i}{i!}
.\end{multline*}
Наконец, подставим результат в формулу~(\ref{eq:invlap_gen}):
\begin{multline}\label{local:not0:gmn}
    L^{-1}_u \left[ \frac{1}{u^k} \psi_{m,n} \left( \frac{1}{2^{m+1} u} - n \right) \right] (t) 
=\\=
    \left( 1 - n^2 \right)  L^{-1}_u \left[ \frac{1}{u^k} r_{m,n}(u) \right](t) +
    \left( \frac{n}{2^m} \right)  L^{-1}_u \left[ \frac{1}{u^{k+1}} r_{m,n}(u) \right](t)
+\\+
    \left( \frac{1}{4^{m+1}} \right)  L^{-1}_u \left[ \frac{1}{u^{k+2}} r_{m,n}(u) \right](t)
=\\=
    \left( 1-n^2 \right) 
    e^{-\frac{n^2}{2}} 2^{(k-1)(m+1)} \frac{2}{\sqrt{3} \pi^{1/4}}
    \sum_{i=0}^{\infty} \sum_{j=0}^{\infty} \frac{\left( \frac{t}{2^{m+1}} \right) ^{2j+k-1+i} (-2)^{-j}}{(2j+k-1+i)!\,j!} \frac{n^i}{i!}
+\\+
    \left( \frac{n}{2^m} \right) 
    e^{-\frac{n^2}{2}} 2^{k(m+1)} \frac{2}{\sqrt{3} \pi^{1/4}}
    \sum_{i=0}^{\infty} \sum_{j=0}^{\infty} \frac{\left( \frac{t}{2^{m+1}} \right) ^{2j+k+i} (-2)^{-j}}{(2j+k+i)!\,j!} \frac{n^i}{i!}
-\\-
    \left( \frac{1}{4^{m+1}} \right) 
    e^{-\frac{n^2}{2}} 2^{(k+1)(m+1)} \frac{2}{\sqrt{3} \pi^{1/4}}
    \sum_{i=0}^{\infty} \sum_{j=0}^{\infty} \frac{\left( \frac{t}{2^{m+1}} \right) ^{2j+k+1+i} (-2)^{-j}}{(2j+k+1+i)!\,j!} \frac{n^i}{i!}
.\end{multline}
Теперь получим выражение для $g_{m,n}(t)$, подставляя полученное соотношение (\ref{local:not0:gmn}) в выражение~(\ref{eq:gmn-laplace})
\begin{multline*}
    g_{m,n}(t) = 
    \frac{1}{t^{k-1}} \frac{\Gamma(k)}{\sqrt{2^m}} L^{-1}_u \left[ \frac{1}{u^k} \psi_{m,n} \left( \frac{1}{2^{m+1} u} - n \right) \right] (t)
=\\=
    \frac{1}{t^{k-1}} \frac{\Gamma(k)}{\sqrt{2^m}} \left( 1-n^2 \right) 
    e^{-\frac{n^2}{2}} 2^{(k-1)(m+1)} \frac{2}{\sqrt{3} \pi^{1/4}}
    \sum_{i=0}^{\infty} \sum_{j=0}^{\infty} \frac{\left( \frac{t}{2^{m+1}} \right) ^{2j+k-1+i} (-2)^{-j}}{(2j+k-1+i)!\,j!} \frac{n^i}{i!}
+\\+
    \frac{1}{t^{k-1}} \frac{\Gamma(k)}{\sqrt{2^m}} \left( \frac{n}{2^m} \right) 
    e^{-\frac{n^2}{2}} 2^{k(m+1)} \frac{2}{\sqrt{3} \pi^{1/4}}
    \sum_{i=0}^{\infty} \sum_{j=0}^{\infty} \frac{\left( \frac{t}{2^{m+1}} \right) ^{2j+k+i} (-2)^{-j}}{(2j+k+i)!\,j!} \frac{n^i}{i!}
-\\-
    \frac{1}{t^{k-1}} \frac{\Gamma(k)}{\sqrt{2^m}} \left( \frac{1}{4^{m+1}} \right) 
    e^{-\frac{n^2}{2}} 2^{(k+1)(m+1)} \frac{2}{\sqrt{3} \pi^{1/4}}
    \sum_{i=0}^{\infty} \sum_{j=0}^{\infty} \frac{\left( \frac{t}{2^{m+1}} \right) ^{2j+k+1+i} (-2)^{-j}}{(2j+k+1+i)!\,j!} \frac{n^i}{i!}
.\end{multline*}
Упрощая полученное выражение, получим:
\begin{multline*}
    g_{m,n}(t)
=
    \frac{\Gamma(k)}{\sqrt{2^m}} e^{-\frac{n^2}{2}} \frac{2}{\sqrt{3} \pi^{1/4}} \left(
    \left( 1-n^2 \right) 
    \sum_{i=0}^{\infty} \sum_{j=0}^{\infty} \frac{\left( \frac{t}{2^{m+1}} \right) ^{2j+i} (-2)^{-j}}{(2j+k-1+i)!\,j!} \frac{n^i}{i!} \right.
+\\+
    \left( \frac{nt}{2^m} \right) 
    \sum_{i=0}^{\infty} \sum_{j=0}^{\infty} \frac{\left( \frac{t}{2^{m+1}} \right) ^{2j+i} (-2)^{-j}}{(2j+k+i)!\,j!} \frac{n^i}{i!}
-\\- \left.
    \left( \frac{t^2}{4^{m+1}} \right) 
    \sum_{i=0}^{\infty} \sum_{j=0}^{\infty} \frac{\left( \frac{t}{2^{m+1}} \right) ^{2j+i} (-2)^{-j}}{(2j+k+1+i)!\,j!} \frac{n^i}{i!}
    \right)
.\end{multline*}
\end{document}
