\documentclass[../../paper.tex]{subfiles}
\graphicspath{{\subfix{../figs/}}}
\begin{document}

\subsection{Нахождение $L^{-1}_u [\frac{1}{u^k} r_{m,n}(u)](t)$}
Выше мы ввели
\[
    L^{-1}_u \left[ \frac{1}{u^k} r_{m,n}(u) \right](t)
.\]
Подставим обратно $r_{m,n}(u)$:
\[
    L^{-1}_u \left[ \frac{1}{u^k} r_{m,n}(u) \right](t) =
%
    L^{-1}_u \left[ \frac{1}{u^k} \frac{2}{\sqrt{3} \pi^{1/4} } 
    \exp \left( -\frac{1}{2} \left( \frac{1}{2^{m+1}u} - n \right)^2  \right)
    \right](t)
.\]
Раскроем квадрат под экспонентой:
\begin{multline*}
    L^{-1}_u \left[ \frac{1}{u^k} r_{m,n}(u) \right](t)
=\\=%
    L^{-1}_u \left[ \frac{1}{u^k} \frac{2}{\sqrt{3} \pi^{1/4} } 
    \exp \left( -\frac{1}{2}
    \left(\left( n^2 \right) - \frac{1}{u} \left(\frac{n}{2^m}\right) + \frac{1}{u^2} \left(\frac{1}{4^{m+1}}\right)\right)
    \right)
    \right](t)
.\end{multline*}
Сгруппируем $2^{m+1}u$:
\begin{multline*}
    L^{-1}_u \left[ \frac{1}{u^k} r_{m,n}(u) \right](t)
=\\=%
    L^{-1}_u \left[ \frac{2^{k(m+1)}}{\left(2^{m+1}u\right)^k} \frac{2}{\sqrt{3} \pi^{1/4} } 
    \exp \left( -\frac{n^2}{2} + \frac{n}{2^{m+1}u} - \frac{1}{2 \left( 2^{m+1}u \right)^2 }
    \right)
    \right](t)
.\end{multline*}
Вынесем множители, не зависящие от $u$, за $L^{-1}_u$:
\begin{multline*}
    L^{-1}_u \left[ \frac{1}{u^k} r_{m,n}(u) \right](t)
=\\=
    e^{-\frac{n^2}{2}} 2^{k(m+1)} \frac{2}{\sqrt{3} \pi^{1/4}}
    L^{-1}_u \left[ \frac{1}{\left(2^{m+1}u\right)^k}
    \exp \left(\frac{n}{2^{m+1}u} - \frac{1}{2 \left( 2^{m+1}u \right)^2 }
    \right)
    \right](t)
.\end{multline*}
Используя лемму \ref{laplace-var-change} о замене переменной в обратном преобразовании Лапласа, делаем замену $s=2^{m+1}u$:
\[
    L^{-1}_u \left[ \frac{1}{u^k} r_{m,n}(u) \right](t) =
%
    e^{-\frac{n^2}{2}} 2^{(k-1)(m+1)} \frac{2}{\sqrt{3} \pi^{1/4}}
    L^{-1}_s \left[ \frac{1}{s^k}
    \exp \left(\frac{n}{s} - \frac{1}{2 s^2 }
    \right)
    \right] \left( \frac{t}{2^{m+1}} \right)
.\]
\end{document}
