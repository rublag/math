\documentclass[../paper.tex]{subfiles}
\begin{document}
\section{Правильная часть ряда Лорана для $f(s) = e^{as} e^{-1/(2s^2)}$}
\begin{Th*}
    Правильная часть ряда Лорана для функции
\[
    f(z) = e^{az} e^{-1/(2z^2)}
.\]
Равна
\[
    \sum_{k=0}^{\infty} z^k \sum_{m=0}^{\infty} \frac{a^{2m+k} / (-2)^m}{(2m+k)!\,m!}
.\]
\end{Th*}
\begin{proof}
Заменим экспоненты рядами:
\[
    e^{az} e^{-1/(2z^2)} =
%
    \left( \sum_{n=0}^{\infty} \frac{(az)^n}{n!} \right)
    \left( \sum_{m=0}^{\infty} \frac{\left( -1 / \left( 2z^2 \right) \right)^m}{m!}  \right) 
.\]
Обе функции аналитичны в $\CC \setminus \{0\}$, поэтому их ряды сходятся абсолютно. Находим ряд Лорана для $f$, перемножая по Коши эти два ряда:
\[
    e^{az} e^{-1/(2z^2)} =
    \sum_{k=-\infty}^{\infty} z^k \sum_{n-2m=k} \chi(n \ge 0) \chi(m \ge 0) \frac{a^n / (-2)^m}{n!\,m!}
.\]
Тогда правильная часть:
\[
    \sum_{k=0}^{\infty} z^k \sum_{n-2m=k} \chi(n \ge 0) \chi(m \ge 0) \frac{a^n / (-2)^m}{n!\,m!} =
    \sum_{k=0}^{\infty} z^k \sum_{m=0}^{\infty} \frac{a^{2m+k} / (-2)^m}{(2m+k)!\,m!}
.\]
\end{proof}
\end{document}
