\documentclass[../paper.tex]{subfiles}
\begin{document}
\begin{Lem}
\label{laurent-3}
Пусть $k \ge 0$; пусть
\[
	f(z) = e^{az} e^{-1/(2z^2)} e^{b/z}
.\]
Тогда $k$-й член ряда Лорана для $f$ равен
\[
	\sum_{l=0}^{\infty}
	\sum_{m=0}^{\infty} \frac{a^{2m+k+l} (-2)^{-m}}{(2m+k+l)!\,m!}
	\frac{n^l}{l!}
.\]
\end{Lem}
\begin{proof}
Положим
\begin{align*}
	& g(z) = e^{az} e^{-1/(2z^2)}, \\
	& h(z) = e^{b/z}
.\end{align*}
Обе функции аналитичны в $\CC \setminus \{0\}$.
Поэтому их ряды сходятся абсолютно и мы можем умножить ряды по Коши, чтобы получить ряд Лорана для $f$.

У функции $e^{n/z}$ положительная часть нулевая.
Поэтому, согласно лемме \ref{product-series} о правильной части произведения голоморфной функции и функции с нулевой положительной частью, 
нам достаточно знать только правильную часть разложения функции $g$, которую мы нашли в предыдущей лемме \ref{laurent-2}.

Пусть $\{\alpha_n\}_{n=-\infty}^\infty$ --- коэффициенты разложения $g(z)$ в ряд Лорана, $\{\beta_n\}_{n=-\infty}^0$ --- коэффициенты разложения $h(z)$,
а $\{\gamma_n\}_{n=-\infty}^\infty$ --- коэффициенты разложения~$f$.

Приведем формулу $k$-го члена их произведения, где $k \ge 0$:
\[
    \gamma_k =
    \sum_{l=-\infty}^{0} \alpha_{k-l} \beta_l =
    \sum_{l=0}^{\infty} \alpha_{k+l} \beta_{-l}
.\]
Формулу для $\alpha_k$ возьмем из леммы \ref{laurent-2}:
\[
	\alpha_k = \sum_{m=0}^{\infty} \frac{a^{2m+k} / (-2)^{-m}}{(2m+k)!\,m!} \\
.\]
Выпишем формулу для $\beta_{-k}$:
\[
	\beta_{-k} = \frac{n^k}{k!}
.\]

Подставим $\alpha_k$ и $\beta_{-k}$ в формулу для $\gamma_k$:
\[
    \gamma_k =
    \sum_{l=0}^{\infty} \alpha_{k+l} \beta_{-l} =
%
    \sum_{l=0}^{\infty}
    \sum_{m=0}^{\infty} \frac{a^{2m+k+l} / (-2)^{-m}}{(2m+k+l)!\,m!} \\
    \frac{n^l}{l!}
.\]
\end{proof}
\end{document}
