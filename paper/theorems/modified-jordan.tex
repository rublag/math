\documentclass[../paper.tex]{subfile}
\begin{document}
Лемма Жордана позволяет использовать основную теорему о вычетах для интеграла по контуру $(-\infty, \infty)$.
Обратное преобразование Лапласа можно найти, используя интеграл Меллина. Этот интеграл использует контур $(\alpha - i\infty, \alpha + i\infty)$.
Если мы модифицируем лемму Жордана, чтобы она использовала контур в виде левой полуокружности с центром в $\alpha$, то сможем использовать основную теорему о вычетах для вычисления обратного преобразования Лапласа.
%
\begin{Lem}[Модифицированная лемма Жордана]\label{modified-jordan}
Пусть $\alpha$, $t$, $R_0$ --- положительные параметры, 
функция $F(s)$ непрерывна в области 
\[
	G = \{s \mid \Re s \le  \alpha \} \cap \{ s \mid |s-\alpha| \ge  R_0 > 0 \}
.\]
Обозначим через $C_R$ полуокружность $|z-\alpha| = R$ в области $G$.
Пусть выполняется соотношение
\[
	\lim\limits_{R \to \infty} \sup\limits_{s \in C_R} |F(s)| = 0
.\]
Тогда
\[
	\lim_{R \to \infty} \int_{C_R} e^{ts} F(s)ds = 0
.\]
\end{Lem}
\begin{proof}
Для наглядности схематично изобразим контур интегрирования и область $G$:

\begin{figure}[!ht]
	\centering
	\usetikzlibrary{arrows}
\usetikzlibrary{decorations.markings}
\usetikzlibrary{patterns}
%
%
\begin{tikzpicture}
    \begin{axis} [
        axis on top,
        axis lines=middle,
        xmin=-5,
        xmax=5,
        ymin=-5,
        ymax=5,
        unit vector ratio=1 1 1
        ]
%
%        \fill[pattern=north west lines, opacity=0.6] (-10,10) rectangle (2,-10);
%        \fill[white] (2, -3) -- (2, 3) arc(90:270:3) -- cycle;
%        \draw[dashed] (2, -10) -- (2, -3) arc(270:90:3) -- (2,10);
%
%
        \begin{scope}[decoration={
            markings,
            mark=between positions 0 and 1 step 5mm with {\arrow {stealth}}}
            ]
%
            \draw [postaction=decorate] (2,-4) -- (2,4) -- plot [domain=1:3, variable=\t] ({2 + 4*cos(\t * pi / 2 r)}, {4*sin(\t * pi / 2 r)}) -- cycle;
        \end{scope}
        \draw[fill=black] (0, 0) circle (2pt);
    \end{axis}
\end{tikzpicture}
%

	\caption{Контур и область.}
\end{figure}

По условию
\[
	\forall \varepsilon > 0 \ \  \exists R \ \ \forall s \in C_R \quad \left| F(s) \right| = \left| F\left(\alpha + Re^{i\varphi}\right) \right|  < \varepsilon
,\] тогда
\[
	\left|\int_{C_R} e^{ts} F(s) ds\right|
	\le \int_{C_R} \left|e^{ts} F(s)\right| |ds|
	\le \varepsilon \int_{C_R} \left|e^{ts}\right| |ds|
.\]
На полуокружности $C_R$ мы можем представить экспоненту $e^{ts}$ в виде:
\[
	\left|e^{ts}\right|
	= \left|e^{t\left(\alpha+R\cos\varphi+Ri\sin\varphi\right)}\right|
	= e^{t\left(\alpha + R\cos\varphi\right)}
.\]
Подставим полученное представление в интеграл:
\[
	\varepsilon \int_{C_R} \left|e^{st}\right| |ds|
	= \varepsilon \int_{\pi/2}^{\pi/2} e^{\alpha t + R t \cos \varphi} \left|d\left(Re^{i\varphi}\right)\right|
\]
Упростим полученное выражение:
\begin{multline*}
	\varepsilon \int_{\pi/2}^{\pi/2} e^{\alpha t + R t \cos \varphi} \left|d\left(Re^{i\varphi}\right)\right|
	= \varepsilon \int_{\pi/2}^{\pi/2} \left|e^{\alpha t + R t \cos \varphi} R i e^{i\varphi} \right| d\varphi
	= R\varepsilon e^{\alpha t} \int_{\pi/2}^{\pi/2} e^{R t \cos\varphi} d\varphi
	\\= R\varepsilon e^{\alpha t} \int_{0}^{\pi} e^{R t \cos \left(\varphi + \frac{\pi}{2}\right)} d\varphi
	= R\varepsilon e^{\alpha t} \int_{0}^{\pi} e^{-R t \sin\varphi} d\varphi
	= 2R\varepsilon e^{\alpha t} \int_{0}^{\pi/2} e^{-R t \sin\varphi} d\varphi
.\end{multline*}
На отрезке $[0, \pi/2]$ выполняется неравенство $\sin\varphi \ge (2/\pi) \varphi$.
А значит,
\[
	2R\varepsilon e^{\alpha t} \int_{0}^{\pi/2} e^{-R t \sin\varphi} d\varphi
	\le 2R\varepsilon e^{\alpha t} \int_{0}^{\pi/2} \exp\left(-R t \frac{2}{\pi}\varphi\right) d\varphi
.\]
Полученный интеграл вычисляется напрямую:
\[
	2R\varepsilon e^{\alpha t} \int_{0}^{\pi/2} e^{-R t \frac{2}{\pi}\varphi} d\varphi
	= 2R\varepsilon e^{\alpha t} \left.\left( \frac{1}{-\frac{2Rt}{\pi}} \exp\left(- \frac{2 R t \varphi}{\pi}\right) \right) \right|_0^{\pi/2} 
        = \frac{\pi\varepsilon}{t} e^{\alpha t} \left(1 - e^{- R t} \right)
.\]
Получившаяся функции стремится к нулю при $R \to \infty$.
Отсюда интеграл по дуге стремится к $0$.
\end{proof}
\end{document}
