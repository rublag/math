\section{Правильная часть ряда Лорана для $f(z) = e^{az} e^{-1/(2s^2)} e^{n/s}$}
\begin{Th*}
Пусть $k \ge 0$; пусть
\[
    f(z) = e^{az} e^{-1/(2s^2)} e^{b/z}
.\]
Тогда $k$-й член ряда Лорана для $f$ равен
\[
    \sum_{l=0}^{\infty}
    \sum_{m=0}^{\infty} \frac{a^{2m+k+l} / \left( -2 \right)^m}{(2m+k+l)!\,m!} \\
    \frac{n^l}{l!}
.\]
\end{Th*}
\begin{proof}
Определим
\begin{align*}
    & g(t) = e^{az} e^{-1/(2s^2)} \\
    & h(t) = e^{b/z}
.\end{align*}
Обе функции голоморфны в $\CC \setminus \{0\}$. Поэтому их ряды сходятся абсолютны и мы можем умножить ряды по Коши, чтобы получить ряд Лорана для $f$.

У функции $e^{n/s}$ правильная часть константна.
Поэтому, согласно теореме о правильной части произведения голоморфной функции и функции с константной правильной частью, 
нам достаточно знать только правильную часть разложения функции $g$, которую мы нашли в предыдущей теореме.

Пусть $\{\alpha_n\}_{n=-\infty}^\infty$ --- коэффициенты для $g$, а $\{\beta_n\}_{n=-\infty}^0$ --- коэффициенты для $h$, а $\{\gamma_n\}_{n=-\infty}^\infty$ --- коэффициенты $f$.
Для наглядности, приведем формулу $k$-го члена их произведения, где $k \ge 0$:
\[
    \gamma_k =
    \sum_{l=-\infty}^{0} \alpha_{k-l} \beta_l =
    \sum_{l=0}^{\infty} \alpha_{k+l} \beta_{-l}
.\]
Приведем также формулы для $\alpha_k$ и $\beta_{-k}$:
\begin{align*}
    & \alpha_k = \sum_{m=0}^{\infty} \frac{a^{2m+k} / \left( -2 \right)^m}{(2m+k)!\,m!} \\
    & \beta_{-k} = \frac{n^k}{k!}
.\end{align*}
Подставим $\alpha_k$ и $\beta_{-k}$ в формулу для $\gamma_k$:
\[
    \gamma_k =
    \sum_{l=0}^{\infty} \alpha_{k+l} \beta_{-l} =
%
    \sum_{l=0}^{\infty}
    \sum_{m=0}^{\infty} \frac{a^{2m+k+l} / \left( -2 \right)^m}{(2m+k+l)!\,m!} \\
    \frac{n^l}{l!}
.\]
\end{proof}
