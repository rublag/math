\documentclass[../paper.tex]{subfiles}
\begin{document}
\begin{Lem}[Правильная часть произведения голоморфной функции и функции с нулевой положительной частью]
\label{product-series}
Пусть $f(z)$, $g(z)$ --- аналитические функции,
коэффициента ряда Лорана для $g(z)$ при положительных степенях нулевые,
$\{a_n\}_{n=-\infty}^\infty$ --- коэффициенты разложения в ряд Лорана функции $f(z)$;
$\{b_n\}_{n=-\infty}^{0}$ --- коэффициенты разложения в ряд Лорана функции $g(z)$.

Тогда в правильной части разложения в ряд Лорана произведения $f(z)g(z)$ участвуют только коэффициенты правильной части функции $f(z)$.
При этом сам ряд имеет вид:

\end{Lem}
\begin{proof}
Разложим $f(z)g(z)$ в ряд Лорана:
\[
	f(z)g(z)
	= \left(\sum_{n=-\infty}^{\infty} a_n z^n\right)
		\left(\sum_{m=-\infty}^{0} b_m z^m\right)
	= \sum_{k=-\infty}^{\infty} z^k \sum_{m=-\infty}^{0} a_{k-m} b_m
.\]
Нас интересуют только правильная часть, поэтому рассматриваем коэффициенты при $k \geqslant 0$.
При этом из ряда Лорана функции $f(z)$ используются коэффициенты $k-m$.
Принимая во внимание, что $k \geqslant 0$ и $m \leqslant 0$, получаем, что $k - m \geqslant 0$.
А значит, используется только правильная часть функции $f(z)$.
\end{proof}
\end{document}
