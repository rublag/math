\documentclass[../paper.tex]{subfiles}
\begin{document}
\begin{Lem}[Замена переменной в обратном преобразовании Лапласа]
\label{laplace-var-change}
Пусть существует $L^{-1}_u [f(cu](t)$, $c>0$.
Тогда
\[
    L^{-1}_u \left[ f(cu) \right](t) = L^{-1}_s \left[ \frac{1}{c}f(s) \right]\left(\frac{t}{c}\right)
.\]
\end{Lem}
\begin{proof}
Воспользуемся формулой Меллина (лемма \ref{lemma:mellin}:
\[
	L^{-1}_u\left[f\left(cu\right)\right](t) 
	= \int_{\alpha-i\infty}^{\alpha+i\infty} e^{ut}f(cu)du 
	= \frac{1}{c} \int_{\alpha-i\infty}^{\alpha+i\infty} e^{\left(cu\right) (t/c)} f(cu) d\left(cu\right)
.\]
Произведем замену $s=cu$
\[
	L^{-1}_u\left[f\left(cu\right)\right](t)
	= \frac{1}{c} \int_{c\alpha-i\infty}^{c\alpha+i\infty} e^{s (t/c)} f(s) ds
.\]
Заменим интеграл на обратное преобразование Лапласа
\[
	L^{-1}_u\left[f\left(cu\right)\right](t)
	= L^{-1}_s \left[ \frac{1}{c} f(s) \right]\left( \frac{t}{c} \right)  
.\]
\end{proof}
\end{document}
