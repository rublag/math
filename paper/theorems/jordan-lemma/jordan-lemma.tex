\section {Модифицированная лемма Жордана}
Лемма Жордана позволяет использовать основную теорему о вычетах для интеграла по контуру $(-\infty, \infty)$.
Обратное преобразование Лапласа можно найти, используя интеграл Меллина. Этот интеграл использует контур $(\alpha - i\infty, \alpha + i\infty)$.
Если мы модифицируем лемму Жордана, чтобы она использовала контур в виде левой полуокружности с центром в $\alpha$, то сможем использовать основную теорему о вычетах для вычисления обратного преобразования Лапласа.
%
\begin{Th*}[Модифицированная лемма Жордана] $ $ \\
    Пусть $\alpha$, $t$ > 0; 
    $F(s)$ непрерывна в области $G = \{\Re s \le  \alpha \} \cap \{ |s-\alpha| \ge  R_0 > 0 \}$;
	$C_R$ --- полуокружность $|z-\alpha| = R$ в $G$;
	$\lim\limits_{R \to \infty} \sup\limits_{s \in C_R} |F(s)| = 0$; \\
    Тогда $\lim\limits_{R \to \infty} \int\limits_{C_R} e^{ts} F(s)ds = 0$
\end{Th*}
\begin{proof} $ $ \\
%
\usetikzlibrary{arrows}
\usetikzlibrary{decorations.markings}
\usetikzlibrary{patterns}
%
%
\begin{tikzpicture}
    \begin{axis} [
        axis on top,
        axis lines=middle,
        xmin=-5,
        xmax=5,
        ymin=-5,
        ymax=5,
        unit vector ratio=1 1 1
        ]
%
        \fill[pattern=north west lines, opacity=0.6] (-5,5) rectangle (2,-5);
        \fill[white] (2, -1) -- (2, 1) arc(90:270:1) -- cycle;
        \draw[dashed] (2, -5) -- (2, -1) arc(270:90:1) -- (2,5);
%
%
        \begin{scope}[decoration={
            markings,
            mark=between positions 0 and 1 step 5mm with {\arrow {stealth}}}
            ]
%
            \draw [thick, postaction=decorate] plot [domain=1:3, variable=\t] ({2 + 4*cos(\t * pi / 2 r)}, {4*sin(\t * pi / 2 r)});
        \end{scope}
    \end{axis}
\end{tikzpicture}
%


По условию, $\forall \varepsilon > 0 \ \  \exists R \ \ \forall s \in C_R \quad \left| F(s) \right| = \left| F\left(\alpha + Re^{i\varphi}\right) \right|  < \varepsilon$, тогда
%
\begin{align*}
    & \left|\int_{C_R} e^{ts} F(s) ds\right| \le
      \int_{C_R} \left|e^{ts} F(s)\right| |ds| \le
      \varepsilon \int_{C_R} \left|e^{ts}\right| |ds|  \\
%
    & \text{Рассмотрим на $C_R$: } \left|e^{ts}\right| = \left|e^{t\left(\alpha+R\cos\varphi+Ri\sin\varphi\right)}\right| = e^{t\left(\alpha + R\cos\varphi\right)} \\
%
    & \text{Отсюда: } \varepsilon \int_{C_R} \left|e^{st}\right| |ds| = 
      \varepsilon \int_{\frac{\pi}{2}}^{\frac{3\pi}{2}} e^{\alpha t + R t \cos \varphi} \left|d\left(Re^{i\varphi}\right)\right| = \\
%
    & \varepsilon \int_{\frac{\pi}{2}}^{\frac{3\pi}{2}} \left|e^{\alpha t + R t \cos \varphi} R i e^{i\varphi} \right| d\varphi = 
      R\varepsilon e^{\alpha t} \int_{\frac{\pi}{2}}^{\frac{3\pi}{2}} e^{R t \cos\varphi} d\varphi = 
      R\varepsilon e^{\alpha t} \int_{0}^{\pi} e^{R t \cos \left(\varphi + \frac{\pi}{2}\right)} d\varphi = \\
%
    & R\varepsilon e^{\alpha t} \int_{0}^{\pi} e^{-R t \sin\varphi} d\varphi = 
      2R\varepsilon e^{\alpha t} \int_{0}^{\frac{\pi}{2}} e^{-R t \sin\varphi} d\varphi \\
%
    & \text{На $[0, \frac{\pi}{2}]$ } \sin\varphi \ge \frac{2}{\pi} \varphi \\
%
    & \implies 2R\varepsilon e^{\alpha t} \int_{0}^{\frac{\pi}{2}} e^{-R t \sin\varphi} d\varphi \le
      2R\varepsilon e^{\alpha t} \int_{0}^{\frac{\pi}{2}} e^{-R t \frac{2}{\pi}\varphi} d\varphi = \\
%
    & 2R\varepsilon e^{\alpha t} \left.\left( \frac{1}{-\frac{2Rt}{\pi}}e^{- \frac{2 R t \varphi}{\pi}} \right) \right|_0^{\frac{\pi}{2}} = 
         \frac{\pi\varepsilon}{t} e^{\alpha t} \left(1 - e^{- R t} \right) \xrightarrow[R \to \infty]{} 0
\end{align*}

Отсюда интеграл по дуге стремится к $0$.
\end{proof}
