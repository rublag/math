\newcommand*{\CC}{\mathbb{C}}
\newcommand*{\RR}{\mathbb{R}}
\newcommand*{\ZZ}{\mathds{Z}}
\newcommand*{\NN}{\mathds{N}}
\newcommand*{\QQ}{\mathds{Q}}

\newcommand*{\EV}{\mathbf{E}}
\newcommand*{\Var}{\mathbf{D}}
\newcommand*{\PR}{\mathbf{P}}

\newcommand*{\D}{\bm{D}}
\newcommand*{\R}{\bm{R}}
\newcommand{\BraceThree}{%
	\multirow{3}{*}{$\left\{\begin{array}{@{}l@{}} \null \\ \null \\ \null\end{array}\right.$}%
}

\newcommand{\BraceTwo}{%
	\multirow{2}{*}{$\left\{\begin{array}{@{}l@{}} \null \\ \null\end{array}\right.$}%
}

\renewcommand\qedsymbol{$\blacksquare$}

\theoremstyle{plain}
\newtheorem{Th}{Теорема}
\newtheorem*{Th*}{Теорема}
\newtheorem{Lem}{Лемма}
\newtheorem*{Lem*}{Лемма}
\newtheorem*{St}{Утверждение}
\newtheorem*{Prop}{Свойства}

\theoremstyle{definition}
\newtheorem*{Def}{Определение}
\newtheorem{Probl}{Задача}
\newtheorem*{Ex}{Пример}
\newtheorem*{Nam}{Обозначение}
\newtheorem*{Agr}{Договоренность}

\theoremstyle{remark}
\newtheorem*{Rem}{Замечание}

\renewcommand{\proofname}{Доказательство}
\addto\captionsrussian{
	\renewcommand{\proofname}{Доказательство}
}

\newenvironment{amatrix}[1]{%
	\left(\begin{array}{@{}*{#1}{c}|c@{}}
	}{%
	\end{array}\right)
}

\DeclareMathOperator*{\Res}{Res}
\DeclareMathOperator{\PE}{E}
\DeclareMathOperator{\PP}{P}

\renewcommand{\Re}{\operatorname{Re}}
\renewcommand{\Im}{\operatorname{Im}}
