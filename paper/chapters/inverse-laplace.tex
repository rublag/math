\documentclass[../paper.tex]{subfiles}
\begin{document}
\section{Сведение задачи к вычислению обратного преобразования Лапласа}
Есть случайные величины $X, Y, Z$. Мы не знаем распределение $X$, знаем распределение $Y$ и наблюдаем $Z$. Кроме того, известно, что $Z = XY$, и что все величины непрерывны. Нужно оценить распределение $X$.

Мы будем использовать вейвлет «Mexican hat», потому что он прост и непрерывен. Его формула:
\[
    \psi(t) = \frac{2}{\sqrt{3} \pi^{1 / 4}} (1-t^2)e^{-t^2 / 2}
.\]

Определим элементы фрейма: \\
\[
    \psi_{m,n}(t) = 
    \frac{1}{\sqrt{2^m}} \psi\left( \frac{t}{2^m}-n \right) =
    \frac{1}{\sqrt{2^m} }\frac{2}{\sqrt{3} \pi^{1 / 4}} \left(1-\left( \frac{t}{2^m} - n \right)^2 \right) e^{-\left( \frac{t}{2^m} - n \right)^2 / 2}
.\]

Рассмотрим случай $Y \sim \chi^2_{2k}$; $X > \delta > 0$, абсолютно непрерывен. Плотность $Y$:
\[
    \chi^2_{2k} \sim \frac{1}{2^k} \frac{1}{\Gamma(k)} x^{k-1} e^{-x / 2}
.\]


Будем строить функции $g_{m,n}$ такие, что $\PE g_{m,n}(Z) = \PE \psi_{m,n}(X)$. Заметим, что достаточно выполнения:
\[
    \forall x \in \Im X \quad \PE g_{m,n} (xY) = \psi_{m,n}(x) 
.\]

Заменим мат. ожидание преобразованием Лапласа и раскроем $\psi_{m,n}$:
\[
    \left( \frac{1}{2x} \right)^k \frac{1}{\Gamma(k)} L_z \left[g_{m,n}(z) z^{k-1}\right] \left( \frac{1}{2x} \right) = 
    \left( \frac{1}{\sqrt{2} } \right)^m \psi_{m,n} \left( \frac{x}{2^m} - n \right) 
.\]

Сделаем замену $u = \frac{1}{2x}$:
\[
    u^k \frac{1}{\Gamma(k)} L_z \left[g_{m,n}(z) z^{k-1}\right] \left( u \right) = 
    \left( \frac{1}{\sqrt{2} } \right)^m \psi_{m,n} \left( \frac{1}{2^{m+1} u} - n \right) 
.\]

Используя обратное преобразование Лапласа, найдем $g_{m,n}(t)$:
\begin{gather*}
    u^k \frac{1}{\Gamma(k)} L_z \left[g_{m,n}(z) z^{k-1}\right] \left( u \right) = 
    \left( \frac{1}{\sqrt{2} } \right)^m \psi_{m,n} \left( \frac{1}{2^{m+1} u} - n \right)
\\
    L_z \left[g_{m,n}(z) z^{k-1}\right] \left( u \right) = 
    \frac{\Gamma(k)}{u^k} \left( \frac{1}{\sqrt{2} } \right)^m \psi_{m,n} \left( \frac{1}{2^{m+1} u} - n \right)
\\
    g_{m,n}(t) t^{k-1} = 
    L^{-1}_u \left[ \frac{\Gamma(k)}{u^k} \left( \frac{1}{\sqrt{2} } \right)^m \psi_{m,n} \left( \frac{1}{2^{m+1} u} - n \right) \right] (t)
\\
    g_{m,n}(t) = 
    \frac{1}{t^{k-1}} \frac{\Gamma(k)}{\sqrt{2^m}} L^{-1}_u \left[ \frac{1}{u^k} \psi_{m,n} \left( \frac{1}{2^{m+1} u} - n \right) \right] (t)
.\end{gather*}

Таким образом мы получили выражение для $g_{m,n}(t)$. Далее мы выразим его через ряды, используя формулу Меллина и основную теорему о вычетах.
\end{document}
