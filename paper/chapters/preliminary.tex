\documentclass[../paper.tex]{subfiles}
\begin{document}
\section{Предварительные сведения}
Мы будем рассматривать дискретное вейвлет-преобразование. Сформулируем основные определения и свойства.
\begin{Def}
	Вейвлет --- это функция $\psi(t)$, которая отвечает следующему свойству:
	\[
		\int_0^\infty \frac{ \left|\hat{\psi}(\xi)\right|^2 }{\left|\xi\right|} d \xi
		= \int_{-\infty}^0 \frac{ \left|\hat{\psi}(\xi)\right|^2 }{\left|\xi\right|} d \xi
		< \infty
	,\] где $\hat{\psi}(\xi)$ -- образ фурье $\psi(t)$.
\end{Def}

Из этого свойства следует, что
\[
	\int_{-\infty}^{\infty} \psi(t) dt = 0
.\]

На основе материнского вейвлета $\psi(t)$ строится система вейвлетов
\[
	\psi_{m,n}(t) = \frac{1}{\sqrt{a^m}} \psi\left(\frac{t}{a^m} - n b\right)
,\]
где $m$, $n$ --- целые числа, $a$, $b$ --- произвольные парметры такие, что $a > 1$, $b > 0$.

Вейвлет $\psi(t)$, на основе которого строится система вейвлетов $\psi_{m,n}(t)$, называется материнским вейвлетом.

В дальнейшем мы будем использовать системы вейвлетов с параметрами $a=2$ и $b=1$.
Таким образом, мы будем использовать систему:
\[
	\psi_{m,n}(t) = \frac{1}{\sqrt{2^m}} \psi\left(\frac{t}{2^m} - n\right)
.\]

Зачастую система вейвлетов не является ортогональной, но образует фрейм.
\begin{Def}
	Семейство $\{\varphi_k\}_{k=1}^\infty$ является фреймом в $L_2(\RR)$, если существуют положительные постоянные $A$ и $B$ такие, что
	\[
		\forall f \in L_2(\RR) \quad A \|f\|^2 \leqslant \sum_{k=1}^\infty \left|\left(f, \phi_k\right)\right|^2 \leqslant B \|f\|^2
	.\]
	Постоянные $A$ и $B$ называются нижней и верхней границами фрейма соответственно.

	Если $A = B$, то фрейм называется жестким.
\end{Def}

Приведем лемму об аппроксимации функции элементами фрейма.
\begin{Lem*}
	Если семейство $\{\varphi_k\}_{k=1}^\infty$ образует фрейм, то
	\[
		f = \frac{2}{A+B} \sum_{k=1}^\infty \left(f, \phi_k\right)\phi_k + Rf
	,\] где \[
		\|R\| \leqslant \frac{B-A}{B+A}
	.\]
\end{Lem*}
Очевидно, что если фрейм жесткий, аппроксимация точна, то есть $R = 0$.
Заметим также, что любой ортогональный базис является жестким фреймом.

Мы будем использовать два вейвлета: Mexican hat и вейвлет Мейера. Приведем их материнские функции.

\begin{Def}
	Назовем вейвлетом Mexican hat вейвлет $\psi(t)$, определенной следующим образом:
	\[
		\psi_{\mathrm{MHAT}}(t) = \frac{2}{\sqrt{3}} \pi^{-1/4} \left(1-t^2\right) e^{-t^2/2}
	.\]
\end{Def}
\begin{Def}
	Назовем вейвлетом Мейера вейвлет $\psi(t)$, определенной следующим образом:
	\[
		\psi(t) = \psi_1(t) + \psi_2(t)
	,\] где
	\begin{align*}
		\psi_1(t) &=
			\frac{
				\frac{4}{3\pi}\left(t-\frac{1}{2}\right) \cos\left(\frac{2\pi}{3}\left(t-\frac{1}{2}\right)\right) 
				- \frac{1}{\pi} \sin \left(\frac{4\pi}{3}\left(t-\frac{1}{2}\right)\right)
			}{
				\left(t-\frac{1}{2}\right) - \frac{16}{9} \left(t - \frac{1}{2}\right)^3
			}, \\
		\psi_2(t) &=
			\frac{
				\frac{8}{3\pi}\left(t-\frac{1}{2}\right) \cos\left(\frac{8\pi}{3}\left(t-\frac{1}{2}\right)\right) 
				+ \frac{1}{\pi} \sin \left(\frac{4\pi}{3}\left(t-\frac{1}{2}\right)\right)
			}{
				\left(t-\frac{1}{2}\right) - \frac{64}{9} \left(t - \frac{1}{2}\right)^3
			}.
	\end{align*}
\end{Def}

Система вейвлетов Mexican hat образует фрейм с границами $A=3{,}223$, $B = 3{,}596$. Система вейвлетов Мейера является ортогональной, т. е. образует базис $L_2(\RR)$.

Также нам потребуются преобразование Лапласа и формула Меллина обратного преобразования Лапласа.
\begin{Def}
Преобразованием Лапласа функции $f(t)$ называют функция $L_t [f](s)$, которая задается формулой
\[
	L_t [f](s) = \int_0^\infty f(t) e^{-st} dt
.\]

\begin{Lem*}[Формула Меллина]
Пусть $F(s) = L_s [f](t)$ --- преобразование Лапласа функции $f(t)$. Тогда
\[
	f(t) = \frac{1}{2\pi i} \int_{\alpha - i \infty}^{\alpha + i \infty} e^{ts} ds
,\] где $\alpha$ --- такое действительное число, что контур $(\alpha - i \infty, \alpha + i \infty)$ лежит правее всех особенностей $F(s)$, и $F(s)$ ограничена на этом контуре.
\end{Lem*}
\end{Def}
\end{document}
