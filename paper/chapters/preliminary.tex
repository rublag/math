\documentclass[../paper.tex]{subfiles}
\begin{document}
\section{Предварительные сведения}
Мы будем рассматривать дискретное вейвлет-преобразование. Сформулируем основные определения и свойства.
\begin{Def}
Материнский вейвлет --- это функция $\psi(t)$, которая отвечает следующему свойству:
\[
	\int_0^\infty \frac{ \left|\hat{\psi}(\xi)\right|^2 }{\left|\xi\right|} d \xi
	= \int_{-\infty}^0 \frac{ \left|\hat{\psi}(\xi)\right|^2 }{\left|\xi\right|} d \xi
	< \infty
,\] где $\hat{\psi}(\xi)$ -- образ фурье $\psi(\xi)$.
\end{Def}

Из этого свойства следует, что
\[
	\int_{-\infty}^{\infty} \psi(t) dt = 0
.\]

Из материнского вейвлета $\psi(t)$ строится система вейвлетов
\[
	\psi_{m,n}(t) = \frac{1}{\sqrt{a^m}} \psi\left(\frac{t}{a^m} - n b\right)
.\]

В дальнейшем мы будем испольвать вейвлеты с $a=2$ и $b=1$.
Таким образом, мы будем использовать систему:
\[
	\psi_{m,n}(t) = \frac{1}{\sqrt{2^m}} \psi\left(\frac{t}{2^m} - n\right)
.\]

Часто система вейвлетов не является ортогональной, но образует фрейм.
\begin{Def}
Семейство $\phi_k$ является фреймом в $L_2(\RR)$, если существуют постоянные $A$ и $B$ такие, что
\[
\forall f \in L_2(\RR) \quad A \|f\|^2 \leqslant \sum_k \left|\left(f, \phi_k\right)\right|^2 \leqslant B \|f\|^2
\]

\begin{Lem*}
Если семейство $\phi_k$ образует фрейм, то
\[
	f = \frac{2}{A+B} \sum_k \left(f, \phi_k\right)\phi_k + Rf
,\] где \[
	\|R\| \leqslant \frac{B-A}{B+A}
\]

Мы будем использовать два вейвлета: Mexican hat и вейвлет Мейера. Приведем их материнские функции.

\begin{Def}
Материнская функция вейвлета Mexican hat:
\[
	\psi(t) = \frac{2}{\sqrt{3}} \pi^{-1/4} \left(1-t^2\right) e^{-t^2/2}
.\]
\end{Def}
\begin{Def}
Материнская функция вейвлета Мейера:
\[
	\psi(t) = \psi_1(t) + \psi_2(t)
,\] где
\begin{align*}
	\psi_1(t) &=
		\frac{
			\frac{4}{3\pi}\left(t-\frac{1}{2}\right) \cos\left(\frac{2\pi}{3}\left(t-\frac{1}{2}\right)\right) 
			- \frac{1}{\pi} \sin \left(\frac{4\pi}{3}\left(t-\frac{1}{2}\right)\right)
		}{
			\left(t-\frac{1}{2}\right) - \frac{16}{9} \left(t - \frac{1}{2}\right)^3
		}, \\
	\psi_2(t) &=
		\frac{
			\frac{8}{3\pi}\left(t-\frac{1}{2}\right) \cos\left(\frac{8\pi}{3}\left(t-\frac{1}{2}\right)\right) 
			+ \frac{1}{\pi} \sin \left(\frac{4\pi}{3}\left(t-\frac{1}{2}\right)\right)
		}{
			\left(t-\frac{1}{2}\right) - \frac{64}{9} \left(t - \frac{1}{2}\right)^3
		}.
\end{align*}
\end{Def}
\end{Lem*}
\end{Def}

Вейвлет Mexican hat образует фрейм с границами $A=3{,}223$, $B = 3{,}596$, вейвлет Мейера является ортогональным, т. е. образует базис $L_2(\RR)$.

Также нам потребуются преобразование Лапласа и формула Меллина обратного преобразования Лапласа.
\begin{Def}
Преобразование лапласа функции $f(t)$ -- это функция $L_t [f](s)$, которая задается формулой
\[
	L_t [f](s) = \int_0^\infty f(t) e^{-st} dt
.\]

\begin{Lem*}[Формула Меллина]
Пусть $F(s) = L_s [f](t)$ -- преобразование Лапласа функции $f(t)$. Тогда
\[
	f(t) = \frac{1}{2\pi i} \int_{\alpha - i \infty}^{\alpha + i \infty} e^{ts} ds
,\] где $\alpha$ такое, что контур лежит правее всех особенностей $F(s)$ и $F(s)$ ограничена на этом контуре.
\end{Lem*}
\end{Def}
\end{document}
