\documentclass[../paper.tex]{subfiles}
\begin{document}
\section{Вывод}
В работе рассмотрена задача построения оценки распределения размера частиц в коллоидных смесях.

Построена аналитическая оценка плотности коэффициента диффузии с использованием вейвлета Mexican hat.

Предложен способ свести задачу к решению уравнения Фредгольма.
Рассмотрены численные методы решения уравнения Фредгольма: МНК--опти\-ми\-зация с $l_2$--регуляризацией,
аддитивный и мультипликативный итеративные методы, метод градиентного спуска.

Произведено сравнение численных методов построения оценок.

Лучший результат показывает оценка методом МНК--оптимизации с $l_2$--регу\-ляризацией.
Метод градиентного спуска позволяет не вычислять заранее матрицу $K$ и допускает вычисления с помощью графического ускорителя.
Итеративные методы позволяют использовать арифметику произвольной точности.

Дальнейшее улучшение оценок возможно при улучшении качества решения уравнения Фредгольма, уменьшении шага дискретизации или
использовании арифметики произвольной точности.

Также численные методы легко адаптируются для случая, когда случайная величина $Y$ имеет распределение, отличное от хи-квадрат.

В практических применениях построение оценок разбивается на 2 шага. На первом шаге находятся функции $g_{m,n}(Z)$. Это ресурсоемкая и вычислительно трудная задача,
но она не зависит от данных. На втором шаге вычисляется сама оценка. Это вычисление можно представить, используя сложение векторов и покомпонентное умножение векторов.
Каждая из этих операций может быть быстро осуществлена.
\end{document}
