\documentclass[../paper.tex]{subfiles}
\begin{document}
\section{Введение}
В работе мы рассмотрим задачу, которая возникает при исследовании коллоидных примесей в жидкости.

Примеси в исследуемой жидкости --- это движущиеся частицы с размерами порядка $10^{-8}$ м.
Для исследования таких примесей используется анализ траекторий наночастиц.

Схема анализа траекторий наночастиц устроена следующим образом:
\begin{enumerate}
	\item жидкость освещается лазером;
	\item частица, которая попала в луч, рассеивает свет;
	\item рассеянный свет попадает на объектив оптического микроскопа;
	\item видеокамера записывает последовательность оптических изображений;
	\item специальное программное обеспечение по последовательности изображений строит траекторию перемещений частиц.
\end{enumerate}

Траектория движения отдельной частицы является броуновским движением с нулевым сносом и дисперсией $\sigma^2$.

Физически, величина $\sigma^2$ является коэффициентом диффузии, поэтому из соотношения Стокса--Эйнштейна следует, что
\[
	\sigma^2 = \frac{c}{d}
,\] где $d$ --- размер частицы, а $c$ --- некоторая константа.

Наша задача ---  по траекториям частиц оценить распределение их размеров $d$.

Задача оценки размеров наночастиц рассматривалась 
в статьях \cite{Wagner2014}, \cite{Matsuura2018},
где для построения оценок использовали методы отличные от рассмотренных в работе.
\subsection{Постановка задачи}
Мы будем изучать задачу, равносильную оценке распределения размера частиц $d$: оценить распределение случайной величины $\sigma^2 = c/d$.

Рассмотрим $n$ случайно выбранных частиц $E_1, \dots, E_n$. 
Обозначим дисперсии для их движения как $\sigma_1^2, \dots, \sigma_n^2$. 

Для $i$-й частицы у нас есть два $k_i$-мерных вектора перемещений: 
\begin{align*}
	A_i^1, \dots, A_i^{k_i} &\text{ --- по оси } x, \\
	A_i^{k_i+1}, \dots, A_i^{2k_i} &\text{ --- по оси } y
.\end{align*}

Мы сначала рассмотрим только частный случай,
когда все $k_i$ равны $k$, а случайная величина $\sigma_i^2$ имеет абсолютно непрерывное распределение.
Затем в разделе \ref{multilength} обобщим оценку на случай разных длин траекторий, то есть, различных величин $k_i$.

Для этого вместо выборки $A_i^1, \dots, A_i^{2k}$ будем рассматривать
достаточную статистику 
\[
	Z_i = \sum\limits_{j=1}^{2k} \left(A_i^j\right)^2
.\]
Так как случайные величины $A_i^1, \dots, A_i^{2k}$ условно независимы при условии $\sigma_i^2$ и имеют условное
распределение $\mathcal{N}\left(0, \sigma_i^2\right)$, можем представить случайную величину $Z_i$ в виде:
\[
Z_i = \sigma_i^2 Y_i,
\]
где 
\begin{align*}
	Y_i \sim \chi^2_{2k},
	X_i = \sigma_i^2
.\end{align*}
При этом $Y_i$ независимы и не зависят от дисперсии $\sigma_i^2$.

Сформулируем математическую постановку задачи.

\begin{Probl}
	Пусть:
	\begin{enumerate}
		\item $X_1, \dots, X_n$ --- независимые одинаково распределенные непрерывные случайные величины
			с неизвестным распределением и положительным носителем; 
		\item $Y_1, \dots, Y_n$ --- независимые одинаково распределенные случайные величины с распределением $\chi^2_{2k}$;
		\item Случайные величины $X_1, \dots, X_n, Y_1, \dots, Y_n$ независимы в совокупности.
		\item $Z_1, \dots, Z_n$ --- наблюдаемые случайные величины, такие, что:
			\[
				Z_i = X_i Y_i
			.\]
	\end{enumerate}
	Основная цель --- по наблюдениям случайных величин $Z_1, ..., Z_n$ оценить плотность случайной величины $X_1$.
\end{Probl}

\end{document}
