\documentclass[../paper.tex]{subfiles}
\begin{document}
\section{Введение}
В работе мы изучим задачу, которая возникает при исследовании коллоидных примесей в жидкости.

Примеси в исследуемой жидкости --- это движущиеся частицы с размерами порядка $10^{-8}$ м.
Для исследования таких примесей используется анализ траекторий наночастиц.

Жидкость просвечивают лазером, когда луч попадает на чатицу, она рассеивает свет.
К микроскопу присоединена камера, которая фиксирует рассеянный свет.

Получается последовательность изображений. Для каждой частицы эта последовательность является
последовательностью проекций частиц на площадь камеры. Мы можем построить векторы перемещений
частиц в проекции на плоскость камеры по этим снимкам. Для отдельной частицы такие перемещения
образуют броуновское движение с нулевым сносом и дисперсией $\sigma^2 = c/d$, где $c$ --- некоторая
константа, а $d$ --- размер частицы.

Проблема в том, что размер частицы не связан напрямую с размером ее изображения.
Наша задача --- оценить распределение истинных размеров частиц по размерам на снимках.

Будем изучать равносильную задачу: оценить распределение $\sigma^2$.
Рассмотрим $n$ случайно выбранных частиц $E_1, \dots, E_n$. Обозначим дисперсии для
их движения как $\sigma_1^2, \dots, \sigma_n^2$. Для $i$-й частицы у нас есть два
$k(i)$-мерных вектора перемещений: $A_i^1, \dots, A_i^{k(i)}$ по оси $x$ 
и $A_i^{k(i)+1}, \dots, A_i^{2k(i)}$ по оси $y$. Мы будем рассматривать только частный случай,
когда все $k(i)$ равны $k$, а $\sigma_i^2$ непрерывна.

$A_i^1, \dots, A_i^{2k}$ условно независимы при условии $\sigma_i^2$ и имеют условное
распределение $\mathcal{N}\left(0, \sigma_i^2\right)$. 
Дальше вместо выборки $A_i^1, \dots, A_i^{2k}$ будем рассматривать
достаточную статистику $Z_i = \sum\limits_{j=1}^{2k} \left(A_i^j\right)^2$.
Заметим далее, что $Z_i = \sigma_i^2 Y_i$, где $Y_i \sim \chi^2_{2k}$.
При этом, $Y_i$ независимы и не зависят от дисперсии $\sigma_i^2$.

Обозначим $X_i = \sigma_i^2$. Тогда задачу можно сформулировать так:
$X_1, \dots, X_n$ --- независимые одинаково распределенные непрерывные случайные величины
с неизвестным распределением и положительным носителем; 
$Y_1, \dots, Y_n$ --- независимые от них н.о.р. с.в. с распределением $\chi^2_{2k}$;
$Z_1, \dots, Z_n = X_1 Y_1, \dots, X_n Y_n$ --- наблюдаемые случайные величины;
а сама задача --- по наблюдениям $Z_1, ..., Z_n$ оценить распределение $X_1$.
\end{document}
