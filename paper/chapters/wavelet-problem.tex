\documentclass[../paper.tex]{subfiles}
\begin{document}
\section{Преобразование задачи для использования вейвлетов}
Рассмотрим задачу \ref{problem}. Для ее решения будем строить функции $g_{m,n}$ такие, что $\PE g_{m,n}(Z) = \PE \psi_{m,n}(X)$ при $m, n \in \mathbb{Z}$. Заметим, что достаточно выполнения соотношения
\[
    \forall x \in \Im X \quad \PE g_{m,n} (xY) = \psi_{m,n}(x) 
.\]

Таким образом, нам нужно найти функции $g_{m,n}$ такие, что
\begin{equation}\label{eq:int-eq}
	\int_0^\infty g_{m,n}(xy) f_Y(y) dy = \psi_{m,n}(x)
.\end{equation}

Тогда плотность распределения случайной величины $X$ будет аппроксимироваться суммой
\begin{equation}\label{eq:reconstruction}
	\frac{2}{A+B} \sum_{m,n} \PE \psi_{m,n}(X)\psi_{m,n}(x)
	= \frac{2}{A+B} \sum_{m,n} \PE g_{m,n}(Z) \psi_{m,n}(x)
,\end{equation}
где $A$, $B$ --- границы фрейма, образованного системой вейвлетов.
Отметим, что если система вейвлетов образует жесткий фрейм (в частности, ортогональный базис) в $L_2(\RR)$, эта аппроксимация точна.

Получаем оценку, где мы использовали вместо $\PE g_{m,n}(Z)$ оценку, основанную на выборочном среднем:
\begin{equation}\label{eq:estimation}
	\hat{f}_X(x) = \sum_{m,n} \sum_i \frac{g_{m,n}(z_i)}{N} \psi_{m,n}(x)
.\end{equation}
\end{document}
