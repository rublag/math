\documentclass[../paper.tex]{subfiles}
\begin{document}
\section{Преобразование задачи для использования вейвлетов}
Повторим задачу.
Есть независимые одинаково распределенные непрерывные (н. о. р) положительные случайные величины (с. в.) $X_1, \dots, X_N$ с неизвестным распредлением.
Есть н. о. р. с. в. $Y_1, \dots, Y_N$ с распределением $\chi^2(2k)$.
Мы наблюдаем случайные величины $Z_1, \dots, Z_N$, которые задаются как $Z_i = X_i Y_i$.
Нужно оценить распределение $X_1$.

Будем строить функции $g_{m,n}$ такие, что $\PE g_{m,n}(Z) = \PE \psi_{m,n}(X)$. Заметим, что достаточно выполнения:
\[
    \forall x \in \Im X \quad \PE g_{m,n} (xY) = \psi_{m,n}(x) 
.\]

Таким образом, нам нужно найти функции $g_{m,n}$ такие, что
\[
	\int_0^\infty g_{m,n}(xy) f_Y(y) dy = \psi_{m,n}(x)
.\]

Тогда
\[
	f_X(x) 
	\approx \sum_{m,n} \PE \psi_{m,n}(X)\psi_{m,n}(x)
	= \sum_{m,n} \PE g_{m,n}(Z) \psi_{m,n}(x)
,\] причем в случае с ортогональным вейвлетом здесь будет равенство.

Получаем оценку:
\[
	f_X(x) 
	\approx \sum_{m,n} \sum_i \frac{g_{m,n}(z_i)}{N} \psi_{m,n}(x)
\]
\end{document}
