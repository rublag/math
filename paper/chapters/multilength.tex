\documentclass[../paper.tex]{subfiles}
\begin{document}
\section{Обобщение на случай разных длин траекторий}\label{multilength}
Мы строили функции вида:
\[
	\PE g_{m,n}(XY) = \PE \psi_{m,n}(X) = c_{m,n}
\] и находили оценку плотности как
\[
	f_X(x) = c_{m,n} \psi_{m,n}(x)
.\]

Теперь рассмотрим случай, когда длины траекторий могут различаться.
Для каждой длины $k$ построим функции $g_{m,n,k}$, как описано выше,
и построим оценку $f_{X,k}(x)$

Пусть для длины траектории $k$ у нас есть $s_k$ наблюдений. И всего $S$ наблюдений
Тогда оценкой $f_X(x)$ будет:
\[
	\sum_{k=1}^K \frac{s_k f_{X,k}(x)}{S}
.\]

Докажем это. Разложим $f_X$ в ряд по вейвлету:
\[
	f_X(x) = \sum_{m,n} c_{m,n} \psi_{m,n}(x).
\]
Раскроем вейвлет-коэффициенты:
\[
	f_X(x)
	= \sum_{m,n} \PE \psi_{m,n}(X) \psi_{m,n}(x)
.\]
Представим математическое ожидание в виде математического ожидания условного математического ожидания при условии длины траектории:
\[
	f_X(x) 
	= \sum_{m,n} \PE_k \left(\PE \left(\psi_{m,n}(X) \mid k \right) \right ) \psi_{m,n}(x).
\]
Для каждой длины траектории мы построили функции $g_{m,n,k}(XY)$. Заменим $\psi_{m,n}(X)$ на $g_{m,n,k}(XY)$:
\[
	f_X(x) 
	= \sum_{m,n} \PE_k \left(\PE \left(g_{m,n}(XY) \mid k \right) \right ) \psi_{m,n}(x).
\]
По линейности математического ожидания, можем внести сумму внутрь:
\[
  f_X(x) = \PE_k \left( \sum_{m,n} \PE \left(g_{m,n}(XY) | k \right) \psi_{m,n}(x) \right).
\]
Вычислим вейвлет-коэффициенты:
\[
  f_X(x) = \PE_k \left(\sum_{m,n} c_{m,n,k} \psi_{m,n}(x) \right).
\]
Заменим вейвлет-разложение на оригинальную функцию:
\[
  f_X(x) = \PE_k f_{X,k}(x).
\]
Получаем оценку:
\[
  f_X(x) = \sum_{k=1}^K \frac{s_k f_{X,k}(x)}{S}.
\]
\end{document}
