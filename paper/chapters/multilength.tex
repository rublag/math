\documentclass[../paper.tex]{subfiles}
\begin{document}
\chapter{Обобщение на случай разных длин траекторий}
Мы строили функции вида:
\[
  \EV g_{m,n}(XY) = \EV g_{m,n}(X) = c_{m,n}
\] и находили оценку плотности как
\[
  f_X(x) = c_{m,n} \psi_{m,n}(x)
.\]

Теперь рассмотрим случай, когда длины траекторий могут различаться.
Для каждой длины $k$ построим функции $g_{m,n,k}$ как описано выше
и построим оценку $f_{X,k}(x)$

Пусть для длины траектории $k$ у нас есть $s_k$ наблюдений. И всего $S$ наблюдений
Тогда оценкой $f_X(x)$ будет
\[
  \sum_{k=1}^K \frac{s_k f_{X,k}(x)}{S}
.\]

Докажем это. Разложим $f_X$ в ряд по вейвлету:
\[
  f_X(x) = \sum_{m,n} c_{m,n} \psi_{m,n}(x).
\]
Раскроем вейвлет-коэффициенты:
\[
  f_X(x) = \sum_{m,n} \EV \psi_{m,n}(XY) \psi_{m,n}(x).
\]
Представим математическое ожидание в виде математического ожидания условного математического ожидания при условии длины траектории:
\[
  f_X(x) = \sum_{m,n} \EV_k \left(\EV \left(\psi_{m,n}(XY) | k \right) \right ) \psi_{m,n}(x).
\]
По линейности математического ожидания, можем внести сумму внутрь:
\[
  f_X(x) = \EV_k \left( \sum_{m,n} \EV \left(\psi_{m,n}(XY) | k \right) \psi_{m,n}(x) \right).
\]
Вычислим вейвлет-коэффициенты:
\[
  f_X(x) = \EV_k \left(\sum_{m,n} c_{m,n,k} \psi_{m,n}(x) \right).
\]
Заменим вейвлет-разложение на оригинальную функцию:
\[
  f_X(x) = \EV_k f_{X,k}(x).
\]
Получаем оценку:
\[
  f_X(x) = \sum_{k=1}^K \frac{s_k f_{X,k}(x)}{S}.
\]
\end{document}
